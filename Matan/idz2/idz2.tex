% !TEX TS-program = pdflatex
% !TEX encoding = UTF-8 Unicode

% This is a simple template for a LaTeX document using the "article" class.
% See "book", "report", "letter" for other types of document.

\documentclass[10pt]{article} % use larger type; default would be 10pt

\usepackage[utf8]{inputenc} % set input encoding (not needed with XeLaTeX)
\usepackage[russian]{babel}

%%% Examples of Article customizations
% These packages are optional, depending whether you want the features they provide.
% See the LaTeX Companion or other references for full information.

%%% PAGE DIMENSIONS
\usepackage{geometry} % to change the page dimensions
\geometry{a4paper} % or letterpaper (US) or a5paper or....
\geometry{margin=1.75cm} % for example, change the margins to 2 inches all round
% \geometry{landscape} % set up the page for landscape
%   read geometry.pdf for detailed page layout information

\usepackage{graphicx} % support the \includegraphics command and options

% \usepackage[parfill]{parskip} % Activate to begin paragraphs with an empty line rather than an indent

%%% PACKAGES
\usepackage{booktabs} % for much better looking tables
\usepackage{array} % for better arrays (eg matrices) in maths
\usepackage{paralist} % very flexible & customisable lists (eg. enumerate/itemize, etc.)
\usepackage{verbatim} % adds environment for commenting out blocks of text & for better verbatim
\usepackage{subfig} % make it possible to include more than one captioned figure/table in a single float
% These packages are all incorporated in the memoir class to one degree or another...
\usepackage{mathtools}
\usepackage{listings}

%%% HEADERS & FOOTERS
\usepackage{fancyhdr} % This should be set AFTER setting up the page geometry
\pagestyle{fancy} % options: empty , plain , fancy
\renewcommand{\headrulewidth}{0pt} % customise the layout...
\lhead{}\chead{}\rhead{}
\lfoot{}\cfoot{\thepage}\rfoot{}

%%% SECTION TITLE APPEARANCE
\usepackage{sectsty}
\allsectionsfont{\sffamily\mdseries\upshape} % (See the fntguide.pdf for font help)
% (This matches ConTeXt defaults)

%%% ToC (table of contents) APPEARANCE
\usepackage[nottoc,notlof,notlot]{tocbibind} % Put the bibliography in the ToC
\usepackage[titles,subfigure]{tocloft} % Alter the style of the Table of Contents
\renewcommand{\cftsecfont}{\rmfamily\mdseries\upshape}
\renewcommand{\cftsecpagefont}{\rmfamily\mdseries\upshape} % No bold!

%%% END Article customizations

%%% The "real" document content comes below...

\title{ИДЗ 2}
\author{Держапольский Юрий Витальевич}
\date{} % Activate to display a given date or no date (if empty),
         % otherwise the current date is printed 

\begin{document}
 \begin{large}  
\maketitle


\begin{enumerate}
 \item Докажите, что $ \lim\limits_{n\to\infty}{\frac{n^3}{4^{n^2}}} = 0 $

Рассмотрим сумму с данным членом $ \sum\limits_{n=1}^{\infty}{\frac{n^3}{4^{n^2}}}$

Для исследования сходимости используем признак Коши: $ C_{n} = \sqrt[n]{a_{n}}  $
\[ C_{n} = \sqrt[n]{\frac{n^3}{4^{n^2}}} = \frac{ (\sqrt[n]n)^3}{4^n} \xrightarrow{n\to\infty} 0 < 1 \]

Значит сумма сходится. Из этого следует, что $ \lim\limits_{n\to\infty}{\frac{n^3}{4^{n^2}}} = 0 $.

\item Исследуйте на сходимость $ \sum\limits_{n=1}^{\infty}{\frac{(3n+2)!}{10^n n^2}}$

Воспользуемся признаком д'Аламбера: $ D_{n} = \frac{a_{n+1}}{a_{n}}  $
\[  D_{n} =  \frac{(3n+5)!}{10^{n+1} (n+1)^2} \frac{10^n n^2}{(3n+2)!} = \frac{(3n+5)(3n+4)(3n+3)n^2}{10(n^2 + 2n+1) } \stackrel{n\to\infty}{\sim} 
\frac{27 n^5}{10 n^2 } \xrightarrow{n\to\infty} \infty \]
Предел стремится к бесконечности, значит сумма расходится.

\item Исследуйте на сходимость $ \sum\limits_{n=1}^{\infty}{n \left( \frac{3n-1}{4n+2} \right)^{2n} }$

Воспользуемся признаком Коши.
\[ C_{n} = \sqrt[n]{n \left( \frac{3n-1}{4n+2} \right)^{2n}} = \sqrt[n]{n}  \left( \frac{3n-1}{4n+2} \right)^2  \xrightarrow{n\to\infty} 1 * \left( \frac{3}{4}\right)^2 = \frac{9}{16} < 1 \]
Значит сумма сходится.

\item Исследуйте на сходимость $ \sum\limits_{n=1}^{\infty}{ \frac{1}{(2n+3)\ln^2{(2n+1)}} }$

\[   \frac{1}{(2n+3)\ln^2{(2n+1)}} \stackrel{n\to\infty}{\sim}  \frac{1}{(2n)\ln^2{(2n)}} \]

Воспользуемся известной суммой $  \sum\limits_{n=1}^{\infty}{ \frac{1}{x^\alpha\ln^\beta{x}} }  $. Т.к. в исследуемой сумме $\alpha = 1$, а $\beta > 1$, то она сходится, а значит и изначальная сумма сходится.

\item  Исследуйте на сходимость $ \sum\limits_{n=1}^{\infty}{ (-1)^{n+1} \left( \frac{n}{2n+1} \right)^{n} }$

Исследуем на абсолютную сходимость. Рассмотрим сумму $ \sum\limits_{n=1}^{\infty}{ \left( \frac{n}{2n+1} \right)^{n} }$

Используем признак Коши.
\[ C_{n} = \sqrt[n]{\left( \frac{n}{2n+1} \right)^{n} } = \frac{n}{2n+1} \xrightarrow{n\to\infty} \frac{1}{2} < 1 \]
Значит сумма сходится. Отсюда следует, что изначальная сумма сходится абсолютно.

\item Напишите программу и вычилите сумму ряда с точностью $\varepsilon$ : $ \sum\limits_{n=1}^{\infty}{ (-1)^{n} \frac{2}{n^2(n+3)}  }$

Поскольку это ряд Лейбница, то остаток суммы можно оценить его первым слагаемым:  
\[ \bigg\vert \sum\limits_{k=n+1}^{\infty}{ (-1)^{k} a_{k} } \bigg\vert \le a_{n+1}  \]
%\Rightarrow \bigg\vert \sum\limits_{k=n+1}^{\infty}{ (-1)^{k} \frac{2}{k^2(k+3)} } \bigg\vert \le \frac{2}{(n+1)^2(n+4)}

\end{enumerate}

\begin{verbatim}

#include <iostream>
#include <iomanip>
#include <cmath>

using namespace std;

int main() {
    double eps, sum = 0, an = 0, minus = 1;
    cout << "Enter epsilon: " << endl;
    cin >> eps;
    cout << endl;
    int n = 1;
    an = 2. / (pow(n, 2) * (n + 3));

    while (an >= eps) {
        an = 2. / (pow(n, 2) * (n + 3));
        minus *= -1;
        sum += minus * an;
        n++;
    }

    cout << fixed << setprecision(abs(log10(eps))) << sum << endl;
    return 0;

}

\end{verbatim}

\end{large}  
\end{document}
