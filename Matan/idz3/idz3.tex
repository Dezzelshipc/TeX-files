% !TEX TS-program = pdflatex
% !TEX encoding = UTF-8 Unicode

% This is a simple template for a LaTeX document using the "article" class.
% See "book", "report", "letter" for other types of document.

\documentclass[10pt]{article} % use larger type; default would be 10pt

\usepackage[utf8]{inputenc} % set input encoding (not needed with XeLaTeX)
\usepackage[russian]{babel}

%%% Examples of Article customizations
% These packages are optional, depending whether you want the features they provide.
% See the LaTeX Companion or other references for full information.

%%% PAGE DIMENSIONS
\usepackage{geometry} % to change the page dimensions
\geometry{a4paper} % or letterpaper (US) or a5paper or....
\geometry{margin=1.75cm} % for example, change the margins to 2 inches all round
% \geometry{landscape} % set up the page for landscape
%   read geometry.pdf for detailed page layout information

\usepackage{graphicx} % support the \includegraphics command and options

% \usepackage[parfill]{parskip} % Activate to begin paragraphs with an empty line rather than an indent

\usepackage{amsfonts}
\usepackage{amssymb}
\usepackage{amsmath}
\usepackage{mathtools}
\usepackage{listings}
\usepackage{subfig}

%%% HEADERS & FOOTERS
\usepackage{fancyhdr} % This should be set AFTER setting up the page geometry
\pagestyle{fancy} % options: empty , plain , fancy
\renewcommand{\headrulewidth}{0pt} % customise the layout...
\lhead{}\chead{}\rhead{}
\lfoot{}\cfoot{\thepage}\rfoot{}

%%% SECTION TITLE APPEARANCE
\usepackage{sectsty}
\allsectionsfont{\sffamily\mdseries\upshape} % (See the fntguide.pdf for font help)
% (This matches ConTeXt defaults)

%%% ToC (table of contents) APPEARANCE
\usepackage[nottoc,notlof,notlot]{tocbibind} % Put the bibliography in the ToC
\usepackage[titles,subfigure]{tocloft} % Alter the style of the Table of Contents
\renewcommand{\cftsecfont}{\rmfamily\mdseries\upshape}
\renewcommand{\cftsecpagefont}{\rmfamily\mdseries\upshape} % No bold!

%%% END Article customizations

%%% The "real" document content comes below...

\title{ИДЗ 3}
\author{Держапольский Юрий Витальевич}
\date{} % Activate to display a given date or no date (if empty),
         % otherwise the current date is printed 

\begin{document}
 \begin{large}  
\maketitle


\begin{enumerate}
\item Найдите область сходимости $ \sum\limits_{n=1}^{\infty}\frac{n+1}{3^n}(x^2-4x+6)^n $

Заметим, что $\forall  x \,\, x^2-4x+6 = x^2-4x+4+2 = (x-2)^2 + 2 > 0 $. Воспользуемся признаком Коши.
\[ C_{n} = \sqrt[n]{\frac{n+1}{3^n}(x^2-4x+6)^n  } = \frac{\sqrt[n]{n+1}}{3}(x^2-4x+6) \xrightarrow{n\to\infty} \frac{1}{3}(x^2-4x+6) < 1 \implies \]
\[ \implies (x-2)^2 < 1 \implies 1< x < 3 \]
Значит ряд сходится при $ x \in (1;3) $. Проверим ряд на концах интервала.
\[  \sum_{n=1}^{\infty}\frac{n+1}{3^n} * 3^n = \sum_{n=1}^{\infty}(n+1) \]
Общий член не стремится к нулю, значит ряд расходится.

\textbf{Ответ:} Обасть сходимости ряда: $ x \in (1;3) $

\item Найдите область сходимости $ \sum\limits_{n=1}^{\infty}\frac{(-1)^n}{(x+n)^n} $

Область определения ряда: $ x + n \neq 0 \implies x \neq -n \implies x \in \mathbb{R} \setminus \mathbb{Z}^{-}   $
\[ \frac{1}{(x+n)^n} \stackrel{n\to\infty}{\sim} \frac{1}{n^n} \]
Так как $ \frac{1}{n^n} \xrightarrow{n\to\infty} 0 $, значит это ряд Лейбница и он сходится.

\textbf{Ответ:} Обасть сходимости ряда: $ x \in \mathbb{R} \setminus \mathbb{Z}^{-} $

\item Найдите область сходимости $ \sum\limits_{n=1}^{\infty}\frac{4^n}{n^2}\sin^{2n}x $

Воспользуемся признаком Коши.
\[ C_{n} = \sqrt[n]{\frac{4^n}{n^2}\sin^{2n}x} = \frac{4}{(\sqrt[n]{n})^2} \sin^{2}x \xrightarrow{n\to\infty} 4 \sin^{2}x < 1 \implies \]
\[ \implies -\frac{1}{2} < \sin x < \frac{1}{2} \implies x \in (-\frac{\pi}{6} + \pi k; \frac{\pi}{6} + \pi k), k \in \mathbb{Z} \]
Проверим на концах отрезка: $(\sin^2 x = \frac{1}{4})$

\[ \sum_{n=1}^{\infty}\frac{4^n}{n^2} \left( \frac{1}{4}\right)^n = \sum_{n=1}^{\infty}\frac{1}{n^2} \]
Этот ряд сходится, значит нужно включить обе концевые точки.

\textbf{Ответ:} Обасть сходимости ряда: $ x \in [\frac{\pi}{6} + \pi k; \frac{\pi}{6} + \pi k], k \in \mathbb{Z} $

\item Найдите область сходимости $ \sum\limits_{n=1}^{\infty}2^{3n}x^n \sin \frac{2x}{n} $

Воспользуемся признаком Коши.
\[ C_{n} = \sqrt[n]{ 2^{3n} |x|^n \left| \sin \frac{2x}{n} \right| } = 2^3 |x| \sqrt[n]{\left| \sin \frac{2x}{n} \right|} 
\xrightarrow{n\to\infty} 8 |x| < 1 \implies -\frac{1}{8} < x < \frac{1}{8} \]
Проверим на концах отрезка:
\[ x = -\frac{1}{8}: \sum_{n=1}^{\infty} \left( 8^{n} \left( -\frac{1}{8} \right)^n \sin \left( -\frac{2}{8n} \right) \right) = \sum_{n=1}^{\infty} \left( (-1)^{n+1} \sin \frac{1}{4n} \right)  \]
Так как $ \sin(\frac{1}{4n}) \xrightarrow{n\to\infty} 0 $, то это ряд Лейбница и он сходится.
\[ x = \frac{1}{8}: \sum_{n=1}^{\infty} \left( 8^{n} \left( \frac{1}{8} \right)^n \sin \left( \frac{2}{8n} \right) \right) = \sum_{n=1}^{\infty} \sin \frac{1}{4n} \implies
 \sin \frac{1}{4n} \stackrel{n\to\infty}{\sim} \frac{1}{4n} \]
Ряд с данным членом расходится, значит и изначальный ряд тоже.

\textbf{Ответ:} Обасть сходимости ряда: $ x \in [ -\frac{1}{8}; \frac{1}{8} ) $

\item Найдите область сходимости $ \sum\limits_{n=1}^{\infty} n^{\sqrt{x}} \arcsin \frac{x}{3^{nx}} $

Область определения ряда:
$
\begin{cases}
    x \geq 0 \\
    -1 \leq \frac{x}{3^{nx}} \leq 1
\end{cases}
$

Найдём значение на концах интервала для второго неравенства при $x \geq 0: $ $ \frac{x}{3^{nx}} |_{x=0} = 0 $ и $ \frac{x}{3^{nx}} \xrightarrow{x\to\infty} 0 $.
Найдём экстремумы.
\[ \left( \frac{x}{3^{nx}} \right)^{'} = \frac{3^{nx} - x 3^{nx} n \ln 3}{3^{2nx}} = \frac{1-xn \ln3}{3^{nx}} = 0 \implies x = \frac{1}{n \ln3} \]
Найдём значение в экстремуме.
\[ 0 < \frac{1}{n\ln3} \frac{1}{3^{\frac{1}{\ln3}}} < \frac{1}{n} \leq 1 \, \forall n  \]
Получили, что $ \forall n; x \geq 0 \implies 0 \leq \frac{x}{3^{nx}} < \frac{1}{n}$. Значит второе неравенство области определения выполнено.

Ислледуем общий член.
\[ n^{\sqrt{x}} \arcsin \frac{x}{3^{nx}} \stackrel{n\to\infty}{\sim} n^{\sqrt{x}} \frac{x}{3^{nx}} \]
Воспользуемся признаком Коши для ряда с данным общим членом.
\[ C_n = \sqrt[n]{n^{\sqrt{x}} \frac{x}{3^{nx}}} =
(\sqrt[n]{n})^{\sqrt{x}} \, \frac{x^{\frac{1}{n}}}{3^x} \xrightarrow{n\to\infty} \frac{1}{3^x} < 1 \implies x > 0 \]
Проверим концевую точку $ x = 0 $
\[ \sum_{n=1}^{\infty} n^{\sqrt{0}} \arcsin \frac{0}{3^{0n}} = \sum_{n=1}^{\infty} 0 = 0  \]
Значит ряд сходится при $x = 0$

\textbf{Ответ:} Обасть сходимости ряда: $ x \in [0; +\infty) $

\item Найдите область сходимости $ \sum\limits_{n=1}^{\infty} 2n^2 \sqrt{x-2} \, e^{-\frac{n^2}{(x-1)^3}} $

Область определения ряда: $
\begin{cases}
    x - 2 \geq 0 \\
    x - 1 \neq 0
\end{cases}
\implies x \geq 2  $

Воспользуемся признаком Коши.
\[ C_n = \sqrt[n]{2n^2 \sqrt{x-2} \, e^{-\frac{n^2}{(x-1)^3}} } = \sqrt[n]{2\sqrt{x-2}}(\sqrt[n]{n})^2 e^{-\frac{n}{(x-1)^3}} \xrightarrow{n\to\infty} 0 < 1 \]
Значит, ряд сходится на области определения.

\textbf{Ответ:} Обасть сходимости ряда: $ x \in [2; +\infty) $

\item Найдите область сходимости $ \sum\limits_{n=1}^{\infty} (x+5)^n \tg \frac{1}{3^n} $

Это степенной ряд, центрированный в точке $ x = -5 $. Найдём радиус сходимости по теореме Коши-Адамара.

\[ \rho = \overline{\lim_{n\to\infty}} \sqrt[n]{\tg \frac{1}{3^n}} = \lim_{n\to\infty}\sqrt[n]{ \frac{1}{3^n} } = \frac{1}{3} \implies R = \frac{1}{\rho} = 3 \]
Значит ряд сходится при $ x \in (-8; -2) $. Проверим на концевых точках.
\[ (\pm 3)^n \tg \frac{1}{3^n} \stackrel{n\to\infty}{\sim} (\pm 3)^n \frac{1}{3^n} = (\pm 1)^n \]
Значит, ряд расходится в концевых точках интервала.

\textbf{Ответ:} Обасть сходимости ряда: $ x \in (-8; -2) $

\item Найдите область сходимости $ \sum\limits_{n=1}^{\infty} \frac{2n+3}{(n+1)^5x^{2n}} $

Область определения ряда $ x \neq 0 $.

Воспользуемся признаком Коши.
\[ C_n = \sqrt[n]{ \frac{2n+3}{(n+1)^5x^{2n}} } = \sqrt[n]{\frac{2n+3}{(n+1)^5}} \frac{1}{x^2} \xrightarrow{n\to\infty} \frac{1}{x^2} < 1 \implies x^2 > 1 \]
Значит, ряд сходится при $ x \in (-\infty; -1) \cup (1; +\infty) $. Приверим в точках $ x^2 = 1 $
\[ \frac{2n+3}{(n+1)^5} \stackrel{n\to\infty}{\sim} \frac{2n}{n^5} = \frac{2}{n^4} \]
Этот ряд сходится, значит изначальный ряд сходится в точках $ x = \pm 1 $

\textbf{Ответ:} Обасть сходимости ряда: $ x \in (-\infty; -1] \cup [1; +\infty) $


\end{enumerate}

\begin{verbatim}


\end{verbatim}

\end{large}  
\end{document}
