\begin{center}
    \textbf{Задание на производственную практику}

    студенту Держапольскому Юрию Витальевичу группы Б9121-01.03.02сп
\end{center}
\noindent на тему <<Разработка внешней обработки для системы 1С ДВФУ по замене картриджей>>

\vspace{10pt}

Вопросы подлежащие разработке:\\
Создание внешних обработок для <<1С предприятия>>, запросы в системе 1С, типы данных в 1С, исследование системы 1С ДВФУ по замене картриджей, бизнес-процессы и связанные задачи в 1С, регистрация внешних обработок как дополнительных в системе 1С.

\vspace{10pt}

Основные источники информации и другие материалы, используемые для разработки темы:\\
Документация <<Платформа 1С:Предприятие>>, документ <<ДВФУ. Подсистема учёта картриджей. Руководство пользователя.>>

\begin{tabular}{ p{6cm} p{8cm} }

    Дата выдачи & <<15>> июля 2024 г. \\

    Руководитель & \signatureuser{А.В. Дегтярёва} \\

    Задание получил & \signatureuser{Ю.В. Держапольский} \\

\end{tabular}

\pagebreak

\begin{center}
    \textbf{Дневник студента}

    \begin{tabular}{|c|m{1.7cm}|m{9cm}|m{3cm}|}
        \hline
        Дата & Рабочее место & Краткое содержание выполняемых работ & Отметки руководителя \\
        \hline
        15.07.2024 & ДВФУ & Создание тестовых внешних обработок &  \\
        \hline
        16.07.2024 & ДВФУ & Изучение структуры подсистемы 1С замены картриджей &  \\
        \hline
        17.07.2024 & ДВФУ & Создание формы поиска по бизнес-процессу &  \\
        \hline
        18.07.2024 & ДВФУ & Создание режима поиска по номеру &  \\
        \hline
        19.07.2024 & ДВФУ & Интегрирование в систему 1С &  \\
        \hline
        22.07.2024 & ДВФУ & Получение обратной связи и доработка обработки &  \\
        \hline
        23.07.2024 & - & Написание отчёта и презентации &  \\
        \hline
    \end{tabular}
\end{center}

\vspace{10pt}

\begin{tabular}{ p{8cm} p{8cm} }

    Студент & \signatureuser{Ю.В. Держапольский} \\

    Руководитель практики от ДВФУ & \signatureuser{А.В. Дегтярёва} \\

\end{tabular}


\pagebreak