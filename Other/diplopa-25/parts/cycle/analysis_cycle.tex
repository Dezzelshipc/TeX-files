\subsection{Равновесные состояния}
Аналогично незамкнутой системе, в системе с частичным восстановлением ресурса (\ref{cycle}) при \(Q > 0\) могут существовать \( n \) равновесных состояний типа \(\left[ N_0, N_1, \ldots, N_q, 0, \ldots, 0 \right]\), которые могут быть найдены из уравнений
\begin{equation} \label{cycle_stationary_equations}
    \frac{dN}{dt} = 0 \Rightarrow
    \left\lbrace\begin{split}
        & Q + \sum_{i=1}^{q} a_i m_i N_i = \alpha_0 N_0 N_1, \\
        & \alpha_i N_{i+1} = k_i \alpha_{i-1} N_{i-1} - m_i, \quad i=\overline{1,q}                
    \end{split}\right.
\end{equation}

Поскольку связь \(N_{i-1}\) и \(N_{i+1}\) точно такая же, что и у незамкнутой модели, то значения \(N_i\) также могут быть определены по формулам (\ref{flow_2s}, \ref{flow_2s1}). Остаётся найти явные выражения для \(N_0\) и \( N_1\).

Используем обозначения (\ref{flow_sub}) и введём новые:
\begin{equation} \label{cycle_sub}
    \begin{split}
    & \varphi_s = \sum_{j=1}^{s} a_{2j} m_{2j} H_{2j-1}, \quad 
    \psi_s = \sum_{j=1}^{s} a_{2j-1} m_{2j-1} H_{2j-2}, \\
    & \sigma_i = \sum_{j=1}^{i} a_j m_j f_{j-1} H_{j-1} \quad (H_0 = 1, \, f_0 = 0).
    \end{split}
\end{equation}

\begin{enumerate}
    \item Пусть \(q = 2s\) -- \textit{чётное}. Тогда аналогично шагам для незамкнутой цепи получаем \( N_1 = f_{2s} \).
    Используя первое уравнение в (\ref{cycle_stationary_equations}), будем иметь            
    \begin{equation*}
    \begin{split}
        & Q + \sum\limits_{i=1}^{s} a_{2i-1} m_{2i-1} H_{2i-2}(N_1 - f_{2i-2}) + \sum\limits_{i=1}^{s} a_{2i} m_{2i} H_{2i-1}(N_0-f_{2i-1}) = \alpha_0 N_0 N_1, \\
        & Q + \sum\limits_{i=1}^{s} a_{2i-1} m_{2i-1} H_{2i-2}(f_{2s} - f_{2i-2})  = \alpha_0 N_0 N_1 - \sum\limits_{i=1}^{s} a_{2i} m_{2i} H_{2i-1}(N_0-f_{2i-1}), \\
        & Q + f_{2s} \sum\limits_{i=1}^{s} a_{2i-1} m_{2i-1} H_{2i-2} - \sum\limits_{i=1}^{s} a_{2i-1} m_{2i-1} H_{2i-2} f_{2i-2} = \\
        & = N_0 \left( \alpha_0 f_{2s} - \sum\limits_{i=1}^{s} a_{2i} m_{2i} H_{2i-1} \right) + \sum\limits_{i=1}^{s} a_{2i} m_{2i} H_{2i-1} f_{2i-1},  \\
        & Q + f_{2s} \psi_s - \sigma_{2s} = N_0 \left( \alpha_0 f_{2s} - \varphi_s \right), \\
        & N_0 = \frac{Q + f_{2s} \psi_s - \sigma_{2s}}{\alpha_0 f_{2s} - \varphi_s}.
    \end{split}
    \end{equation*}

    \item Пусть \(q = 2s+1\) -- \textit{нечётное}. Тогда \( N_1 = f_{2s+1} \) и
    \begin{equation*}
        \begin{split}
            & Q + \sum\limits_{i=1}^{s+1} a_{2i-1} m_{2i-1} H_{2i-2}(N_1 - f_{2i-2}) + \sum\limits_{i=1}^{s} a_{2i} m_{2i} H_{2i-1}(N_0-f_{2i-1}) = \alpha_0 N_0 N_1, \\
            & Q + \sum\limits_{i=1}^{s} a_{2i} m_{2i} H_{2i-1}(f_{2s+1} - f_{2i-1})  = \alpha_0 N_0 N_1 - \sum\limits_{i=1}^{s+1} a_{2i-1} m_{2i-1} H_{2i-2}(N_1 - f_{2i-2}), \\
            & Q + f_{2s+1} \sum\limits_{i=1}^{s} a_{2i} m_{2i} H_{2i-1} - \sum\limits_{i=1}^{s} a_{2i} m_{2i} H_{2i-1} f_{2i-1} = \\
            & = N_1 \left( \alpha_0 f_{2s+1} - \sum\limits_{i=1}^{s+1} a_{2i-1} m_{2i-1} H_{2i-2} \right) + \sum\limits_{i=1}^{s+1} a_{2i-1} m_{2i-1} H_{2i-2} f_{2i-2},  \\
            & Q + f_{2s+1} \varphi_s - \sigma_{2s+1} = N_1 \left( \alpha_0 f_{2s+1} - \psi_{s+1} \right), \\
            & N_0 = \frac{Q + f_{2s+1} \varphi_s - \sigma_{2s+1}}{ \alpha_0 f_{2s+1} - \psi_{s+1} }.
        \end{split}
        \end{equation*}
\end{enumerate}

В итоге имеем:
\begin{enumerate}
    \item \(q = 2s\): \begin{equation} \label{cycle_2s_01}
        N_1 = f_{2s}, \quad 
        N_0 = \frac{Q + f_{2s} \psi_s - \sigma_{2s}}{\alpha_0 f_{2s} - \varphi_s}
    \end{equation}
    \item \(q = 2s+1\): \begin{equation} \label{cycle_2s1_01}
        N_0 = f_{2s+1}, \quad 
        N_1 = \frac{Q + f_{2s+1} \varphi_s - \sigma_{2s+1}}{ \alpha_0 f_{2s+1} - \psi_{s+1} }
    \end{equation}
\end{enumerate}

\begin{statement}
    Если в \textbf{замкнутой} трофической цепи длины \(q\) численность \(N_q > 0\), то \(N_i > 0 \, (i=\overline{1,q-1})\).
\end{statement}

\begin{proof}
    Из условия \( N_q > 0 \) и (\ref{cycle_2s_01}, \ref{cycle_2s1_01}) получим неравенства, ограничивающие скорость поступления внешнего ресурса в систему.
    \begin{enumerate}
        \item \( q = 2s \) \begin{equation}  \label{cycle_lower_2s}
            \begin{split}
                & N_q = N_{2s} = H_{2s-1} (N_0 - f_{2s-1}) > 0, \quad 
                \frac{Q + f_{2s} \psi_s - \sigma_{2s}}{\alpha_0 f_{2s} - \varphi_s} > f_{2s-1}, \\
                & Q > \alpha_0 f_{2s-1} f_{2s} - ( \varphi_s f_{2s-1} + f_{2s} \psi_s - \sigma_{2s} ) = \widetilde{Q}^*(q).
            \end{split}
        \end{equation}
        
        \item \( q = 2s+1 \) \begin{equation}  \label{cycle_lower_2s1}
            \begin{split}
                & N_q = N_{2s+1} = H_{2s} (N_1 - f_{2s}) > 0, \quad 
                \frac{Q + f_{2s+1} \varphi_s - \sigma_{2s+1}}{ \alpha_0 f_{2s+1} - \psi_{s+1} } > f_{2s}, \\
                & Q > \alpha_0 f_{2s+1} f_{2s} - ( \psi_{s+1} f_{2s} + f_{2s+1} \varphi_s - \sigma_{2s+1}) = \widetilde{Q}^*(q).
            \end{split}
        \end{equation}
    \end{enumerate}


    Предположим противное: \(\exists p < q : N_p \leq 0\). Возможны 4 варианта: \(p\) и \(q\) одинаковой чётности и разной чётности.

    \begin{enumerate}
        \item Пусть \(q = 2s \) и \( N_0 = \frac{Q + f_{2s} \psi_s - \sigma_{2s}}{\alpha_0 f_{2s} - \varphi_s}, N_1 = f_{2s}\).
        \begin{enumerate}
            \item \(p = 2u \, (u < s)\), тогда из (\ref{flow_2s}) следует, что \( N_p = N_{2u} \leq 0 \), если \(N_0 \leq f_{2u-1}\). Значит 
            \begin{equation*}
                Q \leq f_{2u-1} ( \alpha_0 f_{2s} - \varphi_s ) - (f_{2s} \psi_s - \sigma_{2s}).
            \end{equation*}
            Сравнивая с (\ref{cycle_lower_2s}) получаем
            \begin{equation*}
                \begin{split}
                & f_{2s-1} ( \alpha_0 f_{2s} - \varphi_s ) - f_{2s} \psi_s + \sigma_{2s}
                < Q \leq 
                f_{2u-1} ( \alpha_0 f_{2s} - \varphi_s ) - f_{2s} \psi_s + \sigma_{2s}, \\
                & f_{2s-1} < f_{2u-1}
                \end{split}
            \end{equation*}
            Это невозможно, поскольку \(f_{2s-1}\) монотонно возрастает с ростом \(s\).

            \item \(p = 2u+1 \, (2u < 2s-1)\), тогда из (\ref{flow_2s1}) следует, что \( N_p = N_{2u+1} \leq 0 \) при \(N_1 \leq f_{2u}\), т.е. \(f_{2s} \leq f_{2u} \). Что также невозможно из-за монотонного возрастания \(f_{2s}\) с ростом \(s\). 
        \end{enumerate}

        \item Пусть \( q = 2s+1 \) и \( N_0 = f_{2s+1}, N_1 = \frac{Q + f_{2s+1} \varphi_s - \sigma_{2s+1}}{ \alpha_0 f_{2s+1} - \psi_{s+1} } \).
        \begin{enumerate}
            \item \(p = 2u \, (2u-1 < 2s)\), тогда \( N_p = N_{2u} \leq 0 \) при \(N_0 \leq f_{2u-1}\). Значит \(f_{2s+1} < f_{2u-1} \). 
            
            Это невозможно, поскольку \(f_{2s-1}\) монотонно возрастает с ростом \(s\).

            \item \(p = 2u+1 \, (u < s)\), тогда \( N_p = N_{2u+1} \leq 0 \) при \(N_1 \leq f_{2u} \), т.е. 
            \begin{equation*}
                Q \leq f_{2u} ( \alpha_0 f_{2s+1} - \psi_{s+1} ) - f_{2s+1} \varphi_s + \sigma_{2s+1}
            \end{equation*}
            Сравнивая с (\ref{cycle_lower_2s1}) получаем
            \begin{equation*}
                \begin{split}
                & \left\{ \begin{split}
                    & f_{2s} ( \alpha_0 f_{2s+1}  - \psi_{s+1} ) - f_{2s+1} \varphi_s + \sigma_{2s+1}
                    < Q, \\
                    & Q \leq f_{2u} ( \alpha_0 f_{2s+1} - \psi_{s+1} ) - f_{2s+1} \varphi_s + \sigma_{2s+1},
                \end{split} \right. \\
                & f_{2s} < f_{2u}.
                \end{split}
            \end{equation*}
            Что также невозможно. 
        \end{enumerate}
    \end{enumerate}
\end{proof}

\subsection{Условия существования цепи фиксированной длины}

Линеаризуем систему (\ref{cycle}) для определения устойчивости в окрестности состояния \(N^* = [ N_0, N_1, \dots, N_q, 0, \dots, 0 ]\) .

Получим матрицу, похожую на (\ref{flow_jacobian_small}), вида
\begin{equation} \label{cycle_stability_matrix}
    J = \left\Vert \begin{matrix}
        A^1_q & C \\
        0 & D_{n-q}
    \end{matrix} \right\Vert,
\end{equation}
где
\begin{equation}
    A^1_q = \left\Vert \begin{matrix}
            -b_0  & c_1-d_0&   c_2    &  \dots   & c_q      \\
            b_1  &  0     &  -d_1    &          &  0       \\
                    & \ddots & \ddots   &  \ddots  &          \\
                    &        & b_{q-1}  &     0    & -d_{q-1} \\
                    &   0    &          & b_{q}    &  0  
    \end{matrix} \right\Vert,
    C = \left\Vert \begin{matrix}
        c_{q+1} & c_{q+2} & \dots & c_{n} \\
                &    0    &       &
    \end{matrix} \right\Vert,
\end{equation}
\( c_i = a_i m_i, i = \overline{1,n} \), а остальные обозначения соответствуют (\ref{flow_jacobian_vars}).

Аналогично из (\ref{flow_jacobian_spectrum}) имеем асимптотическую устойчивость системы при
\begin{equation} \label{cycle_nq_upper}
    N_q < \frac{m_{q+1}}{\alpha_q k_{q+1}}.
\end{equation}
и устойчивости матрицы \( A^q_1 \).

Матрица \(A^1_q\) не является якобиевой (трёхдиагональной), поэтому определять её устойчивость нужно определять методами обычной устойчивости, например с помощью характеристического многочлена.

\begin{equation*}
    P_q (\lambda) = \det (A^1_q - \lambda I) = \left| \begin{matrix}
        -b_0 -\lambda & c_1-d_0&   c_2    &  \dots   & c_q      \\
            b_1  & -\lambda &  -d_1    &          &  0       \\
                & \ddots & \ddots   &  \ddots  &          \\
                &        & b_{q-1}  & -\lambda & -d_{q-1} \\
                &   0    &          & b_{q}    & - \lambda
    \end{matrix} \right|
\end{equation*}
Раскладывая определитель сначала по нижней строке, а потом по последнему столбцу получим:
\begin{equation*}
    \begin{split}
    & P_q (\lambda) = -\lambda P_{q-1} (\lambda) - b_q \left| \begin{matrix}
        -b_0 -\lambda & c_1-d_0&   c_2    &  \dots   & c_{q-2}    & c_q \\
            b_1  & -\lambda &  -d_1    &          &  & 0 \\
                & \ddots & \ddots   &  \ddots  &   \\
                &       &   b_{q-3}  & -\lambda & -d_{q-3} &   \\
                &                &    &   b_{q-2}  &   -\lambda  &      \\
                &   0           &   &     &  b_{q-1}  & -d_{q-1}  \\
    \end{matrix} \right| = \\
    & = -\lambda P_{q-1} (\lambda) - b_q( -d_{q-1} ) P_{q-2} (\lambda) -b_q (-1)^{q} c_q \left| \begin{matrix}
            b_1  & -\lambda &  -d_1    &        & 0  \\
                & \ddots & \ddots   &  \ddots  &   \\
                &       &   b_{q-3}  & -\lambda & -d_{q-3}  \\
                &  0            &    &   b_{q-2}  &   -\lambda  \\
            &                &   &  &   b_{q-1}  \\
    \end{matrix} \right| = \\
    & = -\lambda P_{q-1} (\lambda) + b_q d_{q-1} P_{q-2} (\lambda) - (-1)^{q} c_q \prod_{i=1}^{q} b_i.
    \end{split}
\end{equation*}
Учитывая начальные значения характеристического многочлена получаем рекуррентную формулу:
\begin{equation} \label{cycle_reccur}
    \begin{split}
        & P_q (\lambda) = -\lambda P_{q-1} (\lambda) + b_q d_{q-1} P_{q-2} (\lambda) - (-1)^{q} c_q b_1 \cdots b_q, \\
        & P_0 (\lambda) = - b_0 -\lambda, \\
        & P_1 (\lambda) = \lambda^2 + b_0 \lambda + b_1 (d_0 - c_1).
    \end{split}
\end{equation}

Если характеристическое уравнение \( P_q (\lambda) = 0 \) записано в виде
\begin{equation*}
    \lambda^{q+1} + e_q(\lambda) \lambda^q + e_{q-1}(\lambda) \lambda^{q-1} + \dots + e_{1}(\lambda) \lambda + e_0(q) = 0,
\end{equation*}
тогда, используя (\ref{cycle_reccur}), можно выписать рекуррентные  соотношения для коэффициентов \(e_i (q)\):
\begin{equation} \label{cycle_reccur_coeffs}
    \begin{split}
        & e_i (q) = \left\{\begin{split}
        & b_q d_{q-1} e_0(q-2) - (-1)^q c_q b_1 \cdots b_q, && i = 0, \\
        & e_{i-1} (q-1) + b_q d_{q-1} e_i (q-2), && i = \overline{1,q}, \\
        & 1, && i = q+1, \\
        & 0, && i \geq q+2, \\
    \end{split}\right. \\
    & e_{1} (1) = b_0, \\
    & e_{0} (0) = b_0, \\
    & e_{0} (1) = b_1 (d_0 - c_1).
    \end{split}
\end{equation}

Рассмотрим 2 случая.

\begin{enumerate}
\item Пусть \(q = 1\), тогда характеристический многочлен будет выглядеть как
\begin{equation*}
    \lambda^2 + e_1 (1) \lambda + e_0(1) = \lambda^2 + b_0 \lambda + b_1(d_0 - c_1) = 0.
\end{equation*}
Откуда корни уравнения
\begin{equation*}
    \lambda = \frac{-b_0 \pm \sqrt{b_0^2 - 4b_1 (d_0 - c_1)}}{2}.
\end{equation*}
Для устойчивости необходимо \(\lambda < 0\), тогда это условие эквивалентно
\begin{equation*}
    \begin{split}
        & b_0 > \sqrt{b_0^2 - 4b_1 (d_0 - c_1)}, \\
        & 0 < 4b_1 (d_0 - c_1).
    \end{split}
\end{equation*}
Значит, матрица \(A^1_1\) устойчива, если \(c_1 < d_0\), или \(m_1 a_1 < \alpha_0 N_0\). Из \eqref{cycle_2s1_01} имеем \(N_0 = \frac{m_1}{\alpha_0 k_1}\), поэтому условие устойчивости можно записать в виде \(a_1 k_1 < 1\). При этом
\begin{equation*}
    N_1 = \frac{Q + f_{1} \varphi_0 - \sigma_{1}}{ \alpha_0 f_{1} - \psi_{1} } = \frac{Q + \left( \frac{m_1}{\alpha_1} \cdot \frac{\alpha_1}{k_1 \alpha_0} \right) \cdot 0 - 0}{\alpha_0 \cdot \left( \frac{m_1}{\alpha_1} \cdot \frac{\alpha_1}{k_1 \alpha_0} \right) - a_1 m_1} = \frac{Q k_1}{m_1(1 - a_1 k_1)}.
\end{equation*}
Тогда \(N_1 > 0\) только при \(a_1 k_1 < 1\). Неравенство \eqref{cycle_nq_upper} даёт ограничение сверху на скорость поступления ресурса.
\begin{equation*}
    \begin{split}
        & \frac{Q k_1}{m_1(1 - a_1 k_1)} < \frac{m_2}{\alpha_1 k_2}, \\
        & Q < \frac{m_1 m_2}{\alpha_1 k_1 k_2} (1 - a_1 k_1) = \wt{Q}^*(2).        
    \end{split}
\end{equation*}
Таким образом, ограничение скорости поступления ресурса
\begin{equation*}
    0 < Q < \frac{m_1 m_2}{\alpha_1 k_1 k_2} (1 - a_1 k_1)
\end{equation*}
является необходимым и достаточным условием существования устойчивого равновесия типа \([N_0, N_1, 0, \dots, 0]\), то есть существования замкнутой трофической цепи длины \(1\).

\item Пусть \(q=2\), тогда имеем \( \lambda^3 + e_2(2) \lambda^2 + e_1(2) \lambda + e_0(2) = 0 \). По теореме Виета корни связаны соотношениями:
\begin{equation*}
    \begin{split}
        & \lambda_1 + \lambda_2 + \lambda_3 = -e_2(2), \\
        & \lambda_1 \lambda_2 + \lambda_2 \lambda_3 + \lambda_1 \lambda_3 = e_1(2), \\
        & \lambda_1 \lambda_2 \lambda_3 = -e_0(2).
    \end{split}
\end{equation*}
Для устойчивости нужно \(\lambda_i < 0\), значит \( e_0(2), e_1(2), e_2(2) > 0 \). С помощью формул \eqref{cycle_reccur_coeffs} получим
\begin{equation*}
    \begin{split}
        & e_0(2) = b_2 d_{1} e_0(0) - c_2 b_1 b_2 = b_2 ( d_{1} b_0 - c_2 b_1 ) > 0, \\
        & e_1(2) = e_0(1) + b_2 d_1 e_1(0) = b_1 (d_0 - c_1) + b_2 d_1 \cdot 1 > 0, \\
        & e_2(2) = e_1(1) + b_2 d_1 e_2(0) = b_0 + b_2 d_1 \cdot 0 > 0,
    \end{split}
\end{equation*}
Из первого неравенства следует \( \alpha_0 N_1 \cdot \alpha_1 N_1 > a_2 m_2 \cdot k_1 \alpha_0 N_1 \). По \eqref{cycle_2s_01} \(N_1 = f_2 = \frac{m_2}{\alpha_2} \cdot \frac{\alpha_2}{k_2 \alpha_1} = \frac{m_2}{k_2 \alpha_1}\), значит неравенство становится
\begin{equation*}
    \alpha_1 \frac{m_2}{k_2 \alpha_1} > a_2 m_2 k_1, ~~ a_2 k_1 k_2 < 1. 
\end{equation*}
Второе неравенство записывается в виде
\begin{equation*}
    \begin{split}
        & k_1 \alpha_0 N_1 ( \alpha_0 N_0 - a_1 m_1 ) + k_2 \alpha_1 N_2 \cdot \alpha_1 N_1 > 0, \\
        & \alpha_0 N_0 N_1 - a_1 m_1 N_1 + \frac{k_2 \alpha_1^2}{k_1 \alpha_0} N_1 N_2 > 0, \\
        & (\alpha_0 N_0 N_1 - a_1 m_1 N_1 - a_2 m_2 N_2) + N_2 ( a_2 m_2 + \frac{k_2 \alpha_1^2}{k_1 \alpha_0} N_1 ) > 0, \\
        & Q + a_2 m_2 N_2 + \frac{k_2 \alpha_1^2}{k_1 \alpha_0} N_1 N_2 > 0,
    \end{split}
\end{equation*}
т.е. при \(N_2 > 0\) всегда выполняется. Найдём \(N_0\) и \( N_2\). 
\begin{equation*}
    \begin{split}
        N_0 &= \frac{Q + f_2 \psi_1 - \sigma_2}{\alpha_0 f_2 - \vphi_1} = \frac{Q + \frac{m_2}{k_2 \alpha_1} \cdot a_1 m_1 - a_2 m_2 f_1 g_1}{\alpha_0 \frac{m_2}{k_2 \alpha_1} - a_2 m_2 g_1} =\\
        & = \frac{Q + \frac{m_2}{k_2 \alpha_1} \cdot a_1 m_1 - a_2 m_2 \frac{m_1}{\alpha_1}}{\alpha_0 \frac{m_2}{k_2 \alpha_1} - a_2 m_2 \frac{k_1 \alpha_0}{\alpha_1}} = 
        \frac{\alpha_1 k_2 Q +  m_1 m_2 ( a_1- a_2 k_2 )}{\alpha_0 m_2 ( 1- a_2 k_1 k_2 ) }, \\
        N_2 &= H_1 (N_0 - f_1) = \frac{k_1 \alpha_0}{\alpha_1} \cdot \frac{\alpha_1 k_2 Q +  m_1 m_2 ( a_1- a_2 k_2 )}{\alpha_0 m_2 ( 1 - a_2 k_1 k_2 ) } - \frac{m_1}{\alpha_1} =\\
        & = \frac{\alpha_1 k_1 k_2 Q + m_1 m_2 ( a_1 k_1 - a_2 k_1 k_2 )}{\alpha_1 m_2 ( 1- a_2 k_1 k_2 ) } - \frac{m_1 m_2 ( 1- a_2 k_1 k_2 )}{\alpha_1 m_2 ( 1- a_2 k_1 k_2 )} = \\
        & = \frac{\alpha_1 k_1 k_2 Q + m_1 m_2 ( a_1 k_1 - 1 )}{\alpha_1 m_2 ( 1- a_2 k_1 k_2 ) }.
    \end{split}
\end{equation*}
Поскольку \( N_2 > 0 \) при \( Q > \wt{Q}^*(2)\), то и в этом случае из существования стационарного состояния автоматически следует его устойчивость.
\end{enumerate}

Как видно, для ответа на вопрос об устойчивости циклической трофической цепи приходится проводить дополнительное исследование, и заранее не известно будут ли необходимые условия в то же время и достаточными. Хотя и для любого конкретного \(q\) эта задача разрешима. Можем сформулировать слабое утверждение для циклической трофической цепи:
\begin{corollary}
    Необходимым условием существования замкнутой трофической цепи длины \(q\) является ограничение (сверху и снизу) скорости поступления внешнего ресурса в экосистему:
    \begin{equation}
        \widetilde{Q}^*(q) < Q < \widetilde{Q}^*(q+1).
    \end{equation}
\end{corollary}
Где \(\wt{Q}^*(q)\) задаются формулами \eqref{cycle_lower_2s} и \eqref{cycle_lower_2s1}.

В следующих главах с помощью функции Ляпунова будет найдено достаточное условие устойчивости ветвящейся цепи общего вида, которое также включает циклическую цепь. Это неравенство \eqref{split_h_dost} при \(c'_i = 0\).

\subsubsection{Цикл, замкнутый только на первом уровне}
В реальных экосистемах основную массу мёртвой органики образует отмершие листья растений. То есть один из первых уровней цепи. Поэтому предположим, что выражение \( \sum_{i=1}^{n} a_i m_i N_i \) можно представить в виде 
\begin{equation*}
    a_1 m_1 N_1 + \veps \sum_{i=2}^{n} a_i m_i N_i \xrightarrow{\veps \to 0} a_1 m_1 N_1.
\end{equation*}
Тогда в матрице \eqref{cycle_stability_matrix} положим \(c_1 \neq 0, ~~ c_i = 0, i>1\). В этом случае матрица \(A^1_q\) будет принадлежать к классу якобиевых \eqref{flow_jacobian_big} и аналогично устойчивой при положительных переменных, то есть при \(d_0 - c_1 > 0\). Значит, при выполнении неравенства \(c_1 < d_0\) и \eqref{cycle_nq_upper}, существует трофическая цепь длины \(q\).

Поскольку в нашем случае \(c_i = 0 ~ i > 1\), то в формулах \eqref{cycle_sub} положим \(a_1 \neq 0, ~~ a_i = 0, i>1\). Тогда \(\vphi_s = \sigma_2s = \sigma_{2s+1} = 0, ~ \psi_s = \psi_{s+1} = m_1 a_1 \), и при \(q = 2s\)
\begin{equation} \label{cycle1_2s_01}
    \begin{split}
        & N_0 = \frac{Q + m_1 a_1 f_{2s}}{\alpha_0 f_{2s}}, ~~ N_1 = f_{2s}, \\
        & N_q = N_{2s} = H_{2s-1} \left( \frac{Q + m_1 a_1 f_{2s}}{\alpha_0 f_{2s}} - f_{2s-1} \right),
    \end{split}
\end{equation}
а при \(q = 2s+1\)
\begin{equation} \label{cycle1_2s1_01}
    \begin{split}
        & N_0 = f_{2s}, ~~ N_1 = \frac{Q}{\alpha_0 f_{2s} - a_1 m_1}, \\
        & N_q = N_{2s+1} = H_{2s} \left( \frac{Q}{\alpha_0 f_{2s+1} - a_1 m_1} - f_{2s} \right).
    \end{split}
\end{equation}
Условие \( c_1 < d_0 \) эквивалентно \( a_1 m_1 < \alpha_0 N_0 \). Значит
\begin{equation*}
    \begin{matrix}
        a_1 m_1 f_{2s} < Q + a_1 m_1 f_{2s}, & q = 2s, \\
        a_1 m_1 < \alpha_0 f_{2s+1}, & q = 2s+1. \\
    \end{matrix}
\end{equation*}
Очевидно, что при \( Q > 0 \) первое неравенство выполняется. Второе неравенство перепишем в виде \( \alpha_0 f_{2s} f_{2s+1} - a_1 m_1 f_{2s} > 0 \). Но неравенство \eqref{cycle_lower_2s1} для нашего случая становится \( Q > \alpha_0 f_{2s} f_{2s+1} - a_1 m_1 f_{2s} \). Это неравенство необходимо выполняется для положительности \(N_q > 0\). Таким образом, положительность обеспечивает выполнение неравенства \(c_1 < d_0\) для нечётных \(q\). Поэтому остаётся единственное неравенство \eqref{cycle_nq_upper}, при выполнении которого равновесие устойчиво, то есть замкнутая только по первому уровню трофическая цепь длины \(q\) существует.

Используя выражение для \(N_q\) из \eqref{cycle1_2s_01} и \eqref{cycle1_2s1_01}, неравенство \eqref{cycle_nq_upper} запишем в виде

\begin{subequations}
\begin{itemize}
    \item При \(q = 2s \)
        \begin{align}
            \begin{split}
                & N_{2s} = H_{2s-1} \left( \frac{Q + m_1 a_1 f_{2s}}{\alpha_0 f_{2s}} - f_{2s-1} \right) < \frac{m_{2s+1}}{\alpha_{2s} k_{2s+1}}, \\
                & \frac{Q + m_1 a_1 f_{2s}}{\alpha_0 f_{2s}} < f_{2s-1} + \frac{1}{H_{2s-1}} \frac{\mu_{2s+1}}{g_{2s+1}}, \\
                & Q < \alpha f_{2s} f_{2s+1} - a_1 m_1 f_{2s}.
            \end{split}
        \end{align}
    \item При \( q = 2s+1 \)
        \begin{align}
            \begin{split}
                & N_{2s+1} = H_{2s} \left( \frac{Q}{\alpha_0 f_{2s+1} - a_1 m_1} - f_{2s} \right) < \frac{m_{2s+2}}{\alpha_{2s+1} k_{2s+2}}, \\
                & \frac{Q}{\alpha_0 f_{2s+1} - a_1 m_1} < f_{2s} + \frac{1}{H_{2s}} \frac{\mu_{2s+2}}{g_{2s+2}}, \\
                & Q < \alpha_0f_{2s+1} f_{2s+2} - a_1 m_1 f_{2s+2}.
            \end{split}
        \end{align}
\end{itemize}
\end{subequations}

\begin{corollary}
    Необходимым и достаточным условием существования устойчивой трофической цепи длины \(q\) с замыканием лишь по \(1\)-му уровню является ограничение (сверху и снизу) скорости поступления внешнего ресурса в экосистему:
    \begin{equation}
        \wt{Q}^*(q) < Q < \wt{Q}^*(q+1).
    \end{equation}
\end{corollary}
Здесь, согласно \eqref{cycle_lower_2s} и \eqref{cycle_lower_2s1}
\begin{equation*}
    \wt{Q}^*(q) = \begin{cases}
        \alpha_0 f_{q-1} f_q - a_1 m_1 f_q & q = 2s, \\
        \alpha_0 f_{q-1} f_q - a_1 m_1 f_{q-1} & q = 2s + 1.
    \end{cases}
\end{equation*}