\section{Качественная устойчивость}
    Для дальнейшего анализа устойчивости некоторых трофических цепей понадобится определение и критерии свойства под названием \textit{качественная устойчивость}.

    \textit{Вольтеррвоская} модель сообществ \(n\) видов имеет систему вида
    \begin{equation}
        \frac{d N_i}{d t} = N_i \left( \varepsilon_i - \sum_{j=1}^{n} \gamma_{ij} N_j \right), \quad i=\overline{1,n},
    \end{equation}
    где \(\varepsilon_i\) -- скорость естественного прироста или смертности \(i\)-го вида в отсутствие всех остальных видов, а знак и абсолютная величина \(\gamma_{ij} (i \neq j)\) отражают соответственно характер и интенсивность влияния \(j\)-го вида на \(i\)-вид. \(\gamma_{ii}\) -- показатель внутривидового взаимодействия для \(i\)-го вида. Матрицу \(\Gamma = \pares{ \gamma_{ij} } \), отражающую структуру связей сообщества называют \textit{матрицей сообщества}.

    Для описания только характера связей введём \textit{знаковую матрицу} \(S\). Тогда она связана с матрицей сообщества соотношением 
    \[
        S = -\sign \Gamma = \pares{ - \sign \gamma_{ij} }
    \]

    \begin{definition}
        \textbf{Качественная устойчивость сообщества} -- сохранение устойчивости при любых количественных значениях элементов матрицы \(\Gamma = \pares{ \gamma_{ij} }\), сохраняющих лишь тип взаимодействия между каждой парой видов.
    \end{definition}

    Иными словами, качественная устойчивость означает, что сообщество остаётся устойчивым при любых интенсивностях  всех существующих в нем взаимодействий.

    Пусть динамика сообщества \(n\) видов описывается системой уравнений общего вида
    \begin{equation} \label{generic_n}
        \frac{d N_i}{d t} = f_i( \mb{N} ), \quad i = \overline{1,n},
    \end{equation}
    с функциями \(f_i ( \mb{N} )\) допускающими существование равновесия \(N^* > 0\) и линеаризацию в этой точке, то структура соотношений в сообществе может быть определена по матрице системы (\ref{generic_n}), линеаризованной в точке \(N^*\):
    \begin{equation}
        A = \pares{ \left. \frac{\D f_i ( \mb{N} )}{\D N_j} \right|_{N^*} }.
    \end{equation}
    Эта матрица является \textit{матрицей сообщества}. Она описывает характер и интенсивность взаимодействий между видами. Знаковая матрица \(S\) будет равна
    \begin{equation} \label{sign_partial}
        S = \sign A = \pares{ \sign \frac{\D f_i ( \mb{N} )}{\D N_j} }.
    \end{equation}

    Очевидно, что качественная устойчивость является лишь свойством знаковой структуры \(S\) матрицы сообщества \(A\) и на основании (\ref{sign_partial}) может быть сформулирована на языке матриц.
    \begin{definition}
        \textbf{Качественной устойчивостью матрицы} \(A\) (или знак- \hspace{-2pt}устойчивостью) называется устойчивость матрицы \(A\) при любых значениях абсолютных величин её ненулевых элементов.
    \end{definition}
    Иными словами, \(A\) сохраняет устойчивость при любых численных изменениях её элементов, не нарушающих знаковую структуру \(S = \sign A\).

    Если \(A\) не обладает знак-устойчивостью, то в рамках заданной структуры при некотором наборе \({a_{ij}}\) в спектре \(A\) обнаружатся \(\Rez \lambda_i \geq 0\), при этом может существовать такой набор, что матрица окажется устойчивой.

    Для знаковых матриц \(S\) можно указать взаимно однозначное соответствие со \textit{знаковыми ориентированными графами} (далее для краткости ЗОГ). Это получится, если проводить ориентированные рёбра и приписывать им знаки \(+\) или \(-\) по правилу: если вид \(j\) влияет каким-либо образом на вид \(i\), то проводится ребро \(j \to i\) и ему приписывается знак этого влияния.

    Таким образом, условия качественной устойчивости могут формулироваться как в терминах матриц, так и в терминах соответствующих ЗОГ.

\subsection{Необходимые условия}

    Рассмотрим необходимые условия знак-устойчивости матрицы \(A\) \cite{quirk_rupert}:
    \begin{enumerate}
        \item \(a_{ij} a_{ji} \leq 0 \quad \forall i \neq j\); \label{sign_nes_1}
        \item для любой последовательности индексов \(i_1 \neq i_2 \neq i_e \neq \dots i_m \), \(m > 2\) неравенства \(a_{i_1 i_2} \neq 0, a_{i_2 i_3} \neq 0, \dots, a_{i_{m-1} i_m} \neq 0\) влекут \(a_{i_m i_1} = 0\); \label{sign_nes_2}
        \item \(a_{ii} \leq 0 \quad \forall i, \quad \exists i_0 : a_{i_0 i_0} < 0\); \label{sign_nes_3}
        \item существует ненулевой член в разложении \(\det A\). \label{sign_nes_4}
        \item матрица \(A\) является действительной и неразложимой; \label{sign_nes_neraz}
    \end{enumerate}
    С биологической точки зрения эти условия интерпретируются так: (\ref{sign_nes_1}) означает, что в сообществе не должно быть отношений конкуренции или симбиоза. (\ref{sign_nes_3}) означает, что не должно быть самовозрастающих видов и по крайней мере один вид обладает самодемпфированием. Условие (\ref{sign_nes_2}) означает, что ЗОГ сообщества не содержит ориентированных циклов длиной более 2.

    Условие (\ref{sign_nes_4}) формально означает, что есть такая перестановка \(\sigma\) индексов \(1,2,\dots,n\) такая, что произведение элементов \(s_{ij}\) знаковой матрицы \(S = \sign A\) ненулевое:
    \begin{equation} \label{sign_prod}
        s_{1, \sigma(1)} s_{2, \sigma(2)} \cdots s_{n, \sigma(n)} \neq 0.
    \end{equation}

    Известно, что любая перестановка может быть представлена в виде композиции непересекающихся циклов
    \begin{equation*}
        \sigma = c_1 \cdots c_p
    \end{equation*}
    с циклами \(c_j\) длины \(l_j\) такой, что 
    \begin{equation*}
        1 \leq l_j \leq n, \quad \sum_{j=1}^{p} l_j = n.
    \end{equation*}

    Каждому циклу \(c = (i_1, i_2, \dots, i_l)\) длины \(l\) соответствует группа ненулевых сомножителей произведения (\ref{sign_prod}):
    \begin{equation*}
        a_{i_1 i_2} \neq 0, \, a_{i_2 i_3} \neq 0, \, \dots, \, a_{i_{l-1} i_{l}} \neq 0, \, a_{i_l i_1} \neq 0.
    \end{equation*}
    Это соответствует тому, что в ЗОГ вершины \(i_1, \dots, i_l\) соединены в ориентированный цикл. В итоге, условие (\ref{sign_nes_4}) означает, что существует хотя бы одно разбиение ЗОГ на непересекающиеся циклы, сумма длин которых равна \(n\).

    Можно отметить, что учитывая условие (\ref{sign_nes_2}), запрещающее циклы длиннее 2, и (\ref{sign_nes_1}), в ЗОГ качественно устойчивого сообщества можно выделить \(k\) \( \left( 0 \leq k \leq \frac{n}{2} \right)\) пар видов хищник-жертва так, чтобы остальные \(n - 2k\) видов были самодемпфируемыми (являясь циклами длины 1).

    Существенным моментом является условие (\ref{sign_nes_neraz}), которое требует \textit{неразложимость} матрицы. 

    \begin{definition}
        Матрица \(A\) называется \textbf{разложимой}, если некоторой перестановкой её рядов (строк и соответствующих столбцов) она может быть приведена к виду
        \begin{equation}
            A = \pares{ \begin{matrix}
                B & 0 \\
                C & D
            \end{matrix} },
        \end{equation}
        где \(B\) и \(D\) -- квадратные матрицы порядков \(p\) и \(q\) \((p+q = n)\).
    \end{definition}

    Для сообщества неразложимость означает, что в нём нельзя выделить группу \(p\) \( (1 \leq p \leq n)\) видов так, чтобы они не испытывали никакого влияния со стороны остальных \(n-p\) видов. На языке графов это означает, что невозможно выбрать \(p\) вершин так, чтобы ни одна из них не служила концом стрелок, идущих от каких-либо из остальных \(n-p\) вершин. Для матриц это условие требует, чтобы в каждой строке и каждом столбце должен быть хотя бы один ненулевой недиагональный элемент.

    Для примера существенности неразложимости возьмём граф на рис. \ref{example_sign_zog_3}, который соответствует разложимой матрице
    \begin{equation}
        \pares{ \begin{matrix}
            -a & b & c \\
            0 & 0 & -d \\
            0 & e & 0 
        \end{matrix} }, 
        \quad a, b, c, d, e > 0.
    \end{equation}
    Для этого ЗОГ выполняются условия (\ref{sign_nes_1})--(\ref{sign_nes_4}), но он имеет в спектре пару мнимых чисел \(\lambda_{1,2} = \pm i \sqrt{de} ~~ (\lambda_3 = -a)\), т.е. не является устойчивой.
    
    \begin{figure}[H]
        \centering
        \begin{tikzpicture}
    
            \tikzstyle{roundnode} = [draw, circle, text centered];
            \tikzstyle{squarenode} = [draw, regular polygon, regular polygon sides=4, text centered, inner sep=0];
            \tikzstyle{arrow} = [thick, -{Stealth[length=4mm]}];
            \tikzstyle{arrow2} = [thick, {Stealth[length=4mm]}-{Stealth[length=4mm]} ];

            \node[roundnode] (1) at (0,1.5) {$1$};
            \node[roundnode] (2) at (3,3) {$2$};
            \node[roundnode] (3) at (3,0) {$3$};

            \draw [arrow2] (2) -- node[pos=0.1, anchor=west] {$-$} node[pos=0.9, anchor=west] {$+$} (3);
            \draw [arrow] (3) -- node[pos=0.9, anchor=north] {$+$} (1);
            \draw [arrow] (2) -- node[pos=0.9, anchor=south] {$+$} (1);

            \draw[arrow] (1) edge [out=150, in=-150, looseness=8] node [pos=0.9, anchor=north] {$-$} (1);
    
        \end{tikzpicture}
        \caption{Самолимитируемый вид-комменсал \(1\) (питается другим видом без вреда) связан с парой хищник--жертва \(3\)--\(2\).} \label{example_sign_zog_3}
    \end{figure}

    Условие неразложимости ещё более сужает разнообразие видовых соотношений. Этот факт получается из следующей леммы. Для этого введём понятия: матрица \(A\) обладает \textit{симметричной структурой}, если \(\forall i \neq j : a_{ij} \neq 0 \Rightarrow a_{ji} \neq 0 \), и \textit{ассиметричной структурой}, если \(\exists i \neq j : a_{ij} = 0, a_{ji} \neq 0 \).

    \begin{lemma}
        Если \(A\) удовлетворяет условию (\ref{sign_nes_2}) и обладает асимметричной структурой, то \(A\) разложима. \cite{svilog}
    \end{lemma}

    Из этой леммы и условия (\ref{sign_nes_1}) следует, что симметричные ненулевые элементы неразложимой знак-устойчивой матрицы \(A\) должны иметь противоположные знаки, т.е. единственным типом межвидовых отношений в качественно устойчивом сообществе с неразложимой матрицей могут быть лишь отношения хищник--жертва. 

    Рассмотрим сообщество из 5 видов, ЗОГ которого изображён на рис. \ref{example_sign_unstable_zog}, а матрица выглядит так:
    \begin{equation}
        A = \pares{ \begin{matrix}
            0 & 1 & 0 & 0 & 0 \\
            -1& 0 & 1 & 0 & 0 \\
            0 &-1 &-1 & 1 & 0 \\
            0 & 0 &-1 & 0 & 1 \\
            0 & 0 & 0 &-1 & 0
        \end{matrix} }.
    \end{equation}
    
    Как легко убедиться, матрица \(A\) удовлетворяет всем условиям (\ref{sign_nes_1})--(\ref{sign_nes_4}), и, как показывает граф на рисунке \ref{example_sign_unstable_zog}, является неразложимой (\ref{sign_nes_neraz}).

    \begin{figure}[H]
        \centering
        \begin{tikzpicture}
    
            \tikzstyle{roundnode} = [draw, circle, text centered];
            \tikzstyle{squarenode} = [draw, regular polygon, regular polygon sides=4, text centered, inner sep=0];
            \tikzstyle{arrow} = [thick, -{Stealth[length=4mm]}];
            \tikzstyle{arrow2} = [thick, {Stealth[length=4mm]}-{Stealth[length=4mm]} ];

            \node[roundnode] (1) at (0,0) {$1$};
            \node[roundnode] (2) at (3,0) {$2$};
            \node[roundnode] (3) at (6,0) {$3$};
            \node[roundnode] (4) at (9,0) {$4$};
            \node[roundnode] (5) at (12,0) {$5$};

            \draw [arrow2] (1) -- node[pos=0.1, anchor=north] {$+$} node[pos=0.9, anchor=north] {$-$} (2);
            \draw [arrow2] (2) -- node[pos=0.1, anchor=north] {$+$} node[pos=0.9, anchor=north] {$-$} (3);
            \draw [arrow2] (3) -- node[pos=0.1, anchor=north] {$+$} node[pos=0.9, anchor=north] {$-$} (4);
            \draw [arrow2] (4) -- node[pos=0.1, anchor=north] {$+$} node[pos=0.9, anchor=north] {$-$} (5);
            
            \draw[arrow] (3) edge [in=70, out=110, looseness=10] node [pos=0.9, anchor=west] {$-$} (3);
    
        \end{tikzpicture}
        \caption{ЗОГ сообщества 5 видов: \(i\)-й питается \((i+1)\)-м \((i=\overline{1,4})\), при этом \(3\) вид самолимитируется.} \label{example_sign_unstable_zog}
    \end{figure}

    Однако, спектр \(A\) состоит из чисел \(\lambda_1 \approx -0.36\), \(\lambda_{2,3} \approx -0.32 \pm 1.63 i\), \(\lambda_{4,5} = \pm i\), т.е. содержит чисто мнимые числа. Этот пример показывает, что условия (\ref{sign_nes_1})--(\ref{sign_nes_neraz}) являются лишь необходимыми, но не достаточными условиями знак-устойчивости.


\subsection{Достаточные условия} \label{subsec:sign_dost}
    Для получения достаточных условий можно усилить условие (\ref{sign_nes_3}), описывающее самолимитирующие виды \cite{svilog}.

    Для этого определим понятие <<\textit{хищного сообщества}>>. В ЗОГ заданного сообщества рассмотрим какую-нибудь вершину, включенную в цикл длины $2$ ($2$-цикл), одна из стрелок которого имеет знак \(+\), а другая \(-\). Объединим все вершины, которые связаны с данной вершиной такими \(2\)-циклами. Для новых вершин повторяем процедуру объединения с вершинами, связанными с ними теми же \(2\)-циклами. Иными словами, объединим в одно множество все виды, образующие некоторую структуру связей хищник -- жертва. Максимальное множество таких видов будем называть \textbf{хищным сообществом}, содержащим первый вид. Если какой-то вид не связан соотношением \(+ \, -\) ни с какими другими видами, то будем называть его \textit{тривиальным} хищным сообществом.

    ЗОГ на рис. \ref{example_sign_unstable_zog} содержит лишь одно хищное сообщество, включающее все виды. ЗОГ на рис. \ref{example_sign_zog_3} содержит два сообщества: тривиальное \(\{1\}\) и нетривиальное \(\{2,3\}\).

    Разбиению ЗОГ с матрицей \(A = \pares{ a_{ij} } \) на хищные сообщества можно поставить в соответствие матрицу \(\widetilde{A} = \pares{ \widetilde{a}_{ij} } \) по следующему правилу: \(\widetilde{a}_{ij} = a_{ij}\), если ребро \(a_{ij}\) принадлежит некоторому циклу, и \(\widetilde{a}_{ij} = 0\) иначе. Для ЗОГ и матриц, удовлетворяющих условиям (\ref{sign_nes_1}) и (\ref{sign_nes_2}), это означает стирание всех стрелок, связывающих хищные сообщества, а матрица \(A\) приобретает блочно-диагональный вид с блоками, соответствующими отдельным хищным сообществам. Например, для ЗОГ на рис. \ref{example_sign_zog_3} это означает стирание рёбер \(2 \to 1\) и \(3 \to 1\), а матрица \(\widetilde{A}\) принимает вид
    \begin{equation*}
        \widetilde{A} = \pares{ \begin{array}{c|cc}
            -a & 0 & 0 \\ \hline
            0 & 0 & -d \\
            0 & e & 0 
        \end{array} }.
    \end{equation*}

    \begin{lemma} \label{lemma_a_tilde_a}
        Все собственные числа \(A\) и \(\widetilde{A}\) совпадают.
    \end{lemma}

    \begin{proof}
        Пусть элементу \(a_{rs} \neq 0\) соответствует стрелка графа, не принадлежащая никакому циклу (длины больше 1). Это эквивалентно тому, что любое произведение вида
        \begin{equation*}
            a_{rs} a_{sr}, \quad a_{ir} a_{rs} a_{si}, \quad a_{ij} a_{jr} a_{rs} a_{si}, \quad \dots,
        \end{equation*}
        где \(r,s,i,j,dots\) -- различные индексы, обращается в 0 (следует из (\ref{sign_nes_2})).
        Рассмотрим характеристическую матрицу \(\abs{ A - \lambda I } = \abs{ a_{ij} - \delta_{ij} \lambda } \). При её разложении, все члены, содержащие сомножитель \((a_{rs} - \delta_{rs} \lambda)\), исчезнут. То есть, если положить \(a_{rs} = 0\), то значение определителя не изменится. Таким образом,
        \begin{equation*}
            \det \abs{ A - \lambda I } = \det \abs{ \widetilde{A} - \lambda I }.
        \end{equation*}
    \end{proof}

    Значит по устойчивости хищного сообщества можно судить по устойчивости исходного графа. Очевидно, что для устойчивости должно соблюдаться требование 
    \begin{equation*}
        0 \neq \det A = \det \widetilde{A},
    \end{equation*}
    означающее, что все тривиальные хищные сообщества должны обладать самолимитированием.
    
    \begin{theorem} \label{theorem_sign_spectre}
        Если \(A\) удовлетворяет условиям (\ref{sign_nes_1})--(\ref{sign_nes_4}), то \(\Rez \lambda(\widetilde{A}) \leq 0 \), причём кратность значений с нулевой вещественной частью не превосходит \(1\).
    \end{theorem}
    \begin{proof}
        Рассмотрим отдельное хищное сообщество, включающее \(m\) видов, и соответствующую \((m \times m)\)-матрицу \(\widetilde{A}\). Воспользуемся методом Ляпунова для определения устойчивости линейной системы дифференциальных уравнений 
        \begin{equation} \label{sign_hunter_ode}
            \frac{d \mathrm{x}}{dt} = \widetilde{A} \mathrm{x}.
        \end{equation}

        Нужно построить функцию Ляпунова и определить знак её производной по \(t\). Для этого определим \(m\) \textit{положительных чисел} \(\alpha_i\) следующим образом. Положим \(\alpha_1 = 1\). Для каждого \(j\)-го вида, связанного в \(\widetilde{A}\) с \(i\)-м определим соотношение
        \begin{equation} \label{sign_conntcnted_ij}
            \alpha_j a_{ji} = -\alpha_i a_{ij}, \quad i \neq j,
        \end{equation}
        тогда для видов, связанных с \(1\)-м, имеем
        \begin{equation} \label{sign_conntcnted_1}
            \alpha_i = - \frac{a_{1i}}{a_{i1}} > 0
        \end{equation}
        по условию (\ref{sign_nes_1}) и построению хищного сообщества. Поскольку в графе нет замкнутых петель длины больше 2, получим все числа \(\alpha_1, \dots, \alpha_m\).

        Определим функцию
        \begin{equation} \label{sign_lyapunov_func}
            V(x_1, \dots, x_m) = \sum_{i=1}^{m} \alpha_i x_i^2,
        \end{equation}
        где действительный \(m\)-вектор \(\mathrm{x}\) является решением системы (\ref{sign_hunter_ode}). Очевидно, что данная квадратичная форма положительна определена. Найдём её производную на траекториях системы:
        \begin{equation}
            \frac{dV}{dt} = \frac{\D V}{\D \mathrm{x}} \frac{d \mathrm{x}}{dt} = \nabla V \cdot \widetilde{A} \mathrm{x} = \left( 2 \alpha_i x_i \right) \cdot \left( \sum_{j=1}^{m} a_{ij} x_j \right) = 2 \sum_{i=1}^{m} \left( \alpha_i x_i  \sum_{j=1}^{m} a_{ij} x_j \right).
        \end{equation}
        Для каждого слагаемого вида \(\alpha_i a_{ij} x_i x_j\) имеем симметричное \(\alpha_j a_{ji} x_j x_i\) и в силу соотношения (\ref{sign_conntcnted_ij}) они являются противоположными и исчезнут, оставляя только диагональные элементы. Поэтому получаем
        \begin{equation} \label{sign_lyapunov_func_diag}
            \frac{dV}{dt} = 2 \sum_{i=1}^{m} \alpha_i a_{ii} x_i^2 \leq 0,
        \end{equation}
        поскольку \(a_{ii} \leq 0\).

        Функция \(V\) является функцией Ляпунова для нулевого решения системы (\ref{sign_hunter_ode}) поскольку она:
        \begin{enumerate}
            \item непрерывная вместе с частными производными на \( \mbb{R}^m \).
            \item \(V(0, \dots, 0) = 0\);
            \item \( V(x) > 0, x\neq 0 \).
        \end{enumerate}
        Следовательно, по теореме Ляпунова об устойчивости точки равновесия, нулевое решение локально устойчиво. Поэтому \(\Rez \lambda(\widetilde{A}) \leq 0\).
    \end{proof}

    Таким образом, теорема \ref{theorem_sign_spectre} оставляет лишь два возможности для спектра \(\widetilde{A}\): либо все \(\Rez \lambda(\widetilde{A}) < 0\), и нулевое решение асимптотически устойчиво, либо некоторые собственные числа кратности не более \(1\) имеют нулевые вещественные части -- такую ситуацию иногда называют \textit{нейтральной устойчивостью}. Определим условия, которые отделят эти две ситуации.
    
    При \(\det \widetilde{A} \neq 0\) в ситуации нейтральной устойчивости спектр \(\widetilde{A}\) содержит пары чисто мнимых чисел, которым соответствуют синусоидальные (с постоянной амплитудой) слагаемые в общем решении системы (\ref{sign_hunter_ode}). Будем называть компоненты решения \(x_i(t)\), содержащие такие слагаемые, \textit{осциллирующими}.
    
    В структуре хищного сообщества \(\widetilde{A}\) осциллирующие и неосциллирующие виды должны быть расположены специальным образом. 
    \begin{itemize}
        \item Виды \(x_k\) с самолимитированием (\(a_{kk} < 0\)) не могут быть осциллирующими. Это вытекает из того, что периодическому решению системы (\ref{sign_hunter_ode}) соответствует конечная замкнутая траектория в фазовом пространстве и на ней \( \frac{d V}{d t} \equiv 0 \). С учётом (\ref{sign_conntcnted_1}) и (\ref{sign_lyapunov_func_diag}) имеем, что \(x_k \equiv 0\) всюду вдоль периодического решения.
        
        \item Рассмотрим какой-либо осциллирующий вид \(x_i\). Строка системы (\ref{sign_hunter_ode}) этого вида имеет вид
        \begin{equation*}
            \frac{dx_i}{dt} = \sum_{j=1}^{m} a_{ij} x_j.
        \end{equation*}
        В этой строке есть хотя бы один недиагональный ненулевой элемент, который соответствует влиянию другого осциллирующего вида \(x_j\). То есть, осциллирующий вид должен быть связан хотя бы с одним другим осциллирующим видом.

        \item Если неосциллирующий вид связан с некоторым осциллирующим видом, то он с необходимостью имеет связь и с каким-либо другим осциллирующим видом. \cite{svilog}
    \end{itemize}

    Эти требования к структуре нейтрально устойчивого сообщества могут быть формализированы с помощью понятия <<чёрно-белого теста>>. Будем говорить, что хищное сообщество \textit{удовлетворяет чёрно-белому тесту}, если каждая вершина его графа может окрашена в чёрный или белый цвет, что:
    \begin{enumerate}[label={\asbuk*)}, ref=\asbuk*]
        \item все вершины с самолимитированием -- чёрные; \label{black_white_a}
        \item найдётся хотя бы одна белая вершина; \label{black_white_b}
        \item каждая белая вершина связана по крайней мере с одной другой белой вершиной; \label{black_white_c}
        \item каждая чёрная вершина, связанная с белой, связана хотя бы с одной другой белой вершиной. \label{black_white_d}
    \end{enumerate}

        Например, хищное сообщество \(\{1\}\) ЗОГ рис. \ref{example_sign_zog_3} нарушает требование (\ref{black_white_b}), а \(\{ 2, 3 \}\) полностью удовлетворяет тесту. ЗОГ рис. \ref{example_sign_unstable_zog} удовлетворяет тесту, но если переместить самолимитирование в любую другую вершину, тест перестанет выполняться.

        Если хищное сообщество нарушает чёрно-белый тест, то оно не может быть нейтрально устойчивым, и, следовательно, по теореме \ref{theorem_sign_spectre}, оно асимптотически устойчиво. Таким образом, если все хищные сообщества исходного ЗОГ с матрицей \(A\) не удовлетворяют чёрно-белому тесту, то все \(\Rez \lambda(\widetilde{A}) < 0\) и по лемме \ref{lemma_a_tilde_a} матрица \(A\) устойчива.

        В итоге получаем:
        \begin{statement} \label{sign_stab_dost}
            Для качественной устойчивости любой действительной матрицы \(A\) достаточно выполнения совокупности условий (\ref{sign_nes_1}), (\ref{sign_nes_2}), (\ref{sign_dost_3}), (\ref{sign_nes_4}), (\ref{sign_nes_neraz}), где (\labeltext{3'}{sign_dost_3}) требует, чтобы виды с самолимитированием были расположены таким образом, что все его хищные сообщества нарушают чёрно-белый тест (\ref{black_white_a})--(\ref{black_white_d}).
        \end{statement}