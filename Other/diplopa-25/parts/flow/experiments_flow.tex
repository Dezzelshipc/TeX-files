\subsection{Численные эксперименты}
Рассмотрим систему (\ref{flow}) при \(n=3\) со следующими коэффициентами:
\begin{equation} \label{flow_exp1}
    \begin{split}
        & \alpha_0 = 20, ~~ \alpha_1 = 16, ~~ \alpha_2 = 12, ~~ \alpha_3 = 8; \\
        & k_1 = 0.3, ~~ k_2 = 0.2, ~~ k_3 = 0.1; \\
        & m_1 = 4, ~~ m_2 = 3, ~~ m_3 = 2. \\ 
    \end{split}
\end{equation}
Имеем трофические цепи длиной от \(q=1\) до \(q=3\). При заданных значениях параметров имеем данные интервалы, ограничивающие поступление внешнего ресурса \(Q\):
\begin{enumerate}
    \item \(0 < Q < 12.5\);
    \item \(12.5 < Q < 95.83\ldots\);
    \item \( 95.83\ldots < Q\).
\end{enumerate}
Варьируем значение \(Q\) и получаем графики численностей в равновесии.

Для численного решения используем метод Рунге-Кутты \(4\)-порядка с шагом \(h = 0.01\). Начальные значения численностей равны \(2\).

Обозначения: <<Равн\(\{i\}\)>> -- значение точки равновесия, которой соответствует <<Вид\(\{i\}\)>>.

\begin{figure}[H]
    \centering
    \begin{subfigure}[t]{.45\linewidth}
        \centering
        \includegraphics[width=\textwidth]{pictures/exp_flow/exp1_Q0.5.pdf}
        \caption{\(Q = 0.5\)}
    \end{subfigure}
    \begin{subfigure}[t]{.45\linewidth}
            \centering
            \includegraphics[width=\textwidth]{pictures/exp_flow/exp1_Q12.pdf}
            \caption{\(Q = 12\)}
        \end{subfigure}
    \caption{Численности видов системы, при \(Q\) близко к концам первого интервала.}  \label{fig:flow_exp1_q1}
\end{figure}


\begin{figure}[H]
    \centering
    \begin{subfigure}[t]{.45\linewidth}
        \centering
        \includegraphics[width=\textwidth]{pictures/exp_flow/exp1_Q13.pdf}
        \caption{\(Q = 13\)}
    \end{subfigure}
    \begin{subfigure}[t]{.45\linewidth}
            \centering
            \includegraphics[width=\textwidth]{pictures/exp_flow/exp1_Q95.pdf}
            \caption{\(Q = 95\)}
        \end{subfigure}
    \caption{Численности видов системы, при \(Q\) близко к концам второго интервала.}  \label{fig:flow_exp1_q2}
\end{figure}


\begin{figure}[H]
    \centering
    \begin{subfigure}[t]{.45\linewidth}
        \centering
        \includegraphics[width=\textwidth]{pictures/exp_flow/exp1_Q100.pdf}
        \caption{\(Q = 100, \quad N_3 \approx 0.0108\ldots \)}
    \end{subfigure}
    \begin{subfigure}[t]{.45\linewidth}
            \centering
            \includegraphics[width=\textwidth]{pictures/exp_flow/exp1_Q1000.pdf}
            \caption{\(Q = 1000\)}
        \end{subfigure}
    \caption{Численности видов системы, при \(Q\) близко к началу третьего интервала и на некотором отдалении.}  \label{fig:flow_exp1_q3}
\end{figure}

Как видно по графикам, модель ведёт себя в соответствии с теоретическим анализом. На каждом из интервалов ровно нужное количество видов остаётся в живых. 

Для сравнения поведения системы при <<напряжённых>> трофических связях, возьмём систему (\ref{flow_full}) при трофических функциях \( V_i(x) = \alpha_i \arctan x \). Т.е. виды могут насыщаться и не все его жертвы будут становиться добычей.


\begin{figure}[H]
    \centering
    \subfigexp{0.5}{pictures/exp_flow/exp2_Q}{.3}
    \subfigexp{17}{pictures/exp_flow/exp2_Q}{.3}
    \subfigexp{20}{pictures/exp_flow/exp2_Q}{.3}
    \subfigexp{34}{pictures/exp_flow/exp2_Q}{.3}
    \subfigexp{35}{pictures/exp_flow/exp2_Q}{.3}
    \subfigexp{40}{pictures/exp_flow/exp2_Q}{.3}
    \subfigexp{100}{pictures/exp_flow/exp2_Q}{.3}
    \subfigexp{1000}{pictures/exp_flow/exp2_Q}{.3}
\caption{Численности видов системы.}  \label{fig:flow_exp2}
\end{figure}

Как видно, с некоторого значения \(Q\) асимптотическая устойчивость прекращается и появляется периодическая динамика, что свидетельствует о переходе к нейтральной устойчивости.