\secnonum{ВВЕДЕНИЕ}
Многие области науки исследуют взаимодействия между большим многообразием субъектов. В частности, экология занимается исследованием поведения, жизнедеятельности и среды обитания живых существ в нашем мире. Эти существа, как правило, образуют группы с разными способами взаимодействия между ними. Одним из самых частых и наиболее известных способов взаимодействия является структура <<хищник-жертва>>. Однако такая структурная единица сама по себе редко встречается в природе. Обычно между собой взаимодействуют десятки видов. Среди всех таких структур можно выделить структуру под названием <<трофическая цепь>>. Она характеризуется тем, что имеет некоторый поток энергии, приходящий извне, который переходит от одного вида живых существ к другому, а от него к следующему и так далее, подобно цепочке. В природе такие цепочки могут переплетаться между собой и образовывать <<трофические сети>>.

На данный момент темы моделирования трофических сетей пусть и не особо популярны, однако не теряют свою актуальность. Описание сложных процессов взаимодействия биологических сообществ помогает предсказывать динамику в меняющейся окружающей среде, будь то воздействие вымирающих видов, изменение климата или загрязнение мест обитания живых существ. Анализ трофических связей также помогает оптимизировать
управление ресурсами, бороться с вредителями и устойчиво использовать биоресурсы.

Для математического моделирования сообществ живых существ разных видов строятся модели. Самыми известными моделями в экологии являются модели Лотки-Вольтерры. Но они описывают достаточно простые случаи взаимодействия. В то время как описание трофических сетей процесс несколько более сложный вследствие большей комплексности структуры сообщества. Работы, включающие исследования подобных сообществ в основном рассматривают простые случаи структур.

В рамках данной работы проводится исследование динамики и устойчивости структур, являющихся частными случаями <<трофической сети>>: <<трофическая цепь>> и <<ветвящаяся трофическая цепь>>, в рамках моделей <<почти вольтерровского типа>>.

Целью данной работы является получение наиболее полного аналитического описания данных структур и сравнение результатов путём проведения численных экспериментов.
