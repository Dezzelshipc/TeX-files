\section{Введение}
    В экологии структура сообщества, демонстрирующая перенос энергии, заключённой в пище от одного вида к другому, где виды связаны между собой отношениями хищник-жертва, называется \textit{трофической цепью}. При каждом очередном переносе значительная часть энергии (\( \approx \)70-80\%) теряется, расходуясь на дыхание и переходя в тепло. Обычно такие потери энергии ограничивают число <<звеньев>> цепи обычно до четырёх-пяти. В существующие цепи могут занестись извне новые виды особей, которые могли бы образовать следующий трофический уровень. В результате увеличения количества энергии, поступающей в систему, или в результате каких-либо воздействий (например, внесения удобрений) значительно возрастает продуктивность первого уровня, вследствие чего может возникнуть и закрепиться новый трофический уровень, обусловленный имеющимся генерационным материалом.
    
    Трофические цепи обычно не изолированы друг от друга, а переплетаются и образуют трофический граф (трофическую цепь). Примером такой трофической сети может послужить экосистема небольшого ручья\cite{jones_river}, изображённая на рис. \ref{small_river_graph}.
    Это открытая экосистема, часть основного ресурса в которую поступает в виде опавших листьев \textit{1} и других органических остатков \textit{2}, приносимых течением. Она включает три трофических уровня. Виды \textit{3} -- зеленые водоросли и \textit{4} -- диатомовые водоросли -- образуют уровень продуцентов; виды \textit{5} -- веснянка, \textit{6} -- поденки и комары-дергуны, \textit{7} -- ручейники и \textit{8} -- поденка (\textit{Ecdyonurus}) -- уровень первичных консументов, а виды \textit{5} -- веснянка (\textit{Perla}) и \textit{12} -- ручейник (\textit{Dinocras}) -- уровень вторичных консументов. Формы 9 -- ручейники, строящие ловчую сеть, -- и \textit{10} -- ручейник (\textit{Rhyacophila}) -- занимают некоторый промежуточный уровень. Здесь можно выделить много последовательностей видов, образующих трофические цепи, например: \(3 \to 6 \to 12\) или \( 2 \to 7 \to 11 \). 

    \begin{figure}[H]
        \centering
        \begin{tikzpicture}
            \usetikzlibrary{shapes.multipart}
    
            \tikzstyle{roundnode} = [draw, circle, text centered,text width=5mm];
            \tikzstyle{squarenode} = [draw, regular polygon, regular polygon sides=4, text centered, inner sep=0];
            \tikzstyle{arrow} = [thick, -{Stealth[length=4mm]}];
            \tikzstyle{arrow2} = [thick, {Stealth[length=4mm]}-{Stealth[length=4mm]} ];

            \node[roundnode] (1) at (0,0) {$1$};
            \node[roundnode] (2) at (5.5,0) {$2$};
            \node[roundnode] (3) at (1.75,-3) {$3$};
            \node[roundnode] (4) at (9,-3) {$4$};
            \node[roundnode] (5) at (0,-6) {$5$};
            \node[roundnode] (6) at (4.5,-6) {$6$};
            \node[roundnode] (7) at (7.5,-6) {$7$};
            \node[roundnode] (8) at (10,-6) {$8$};
            \node[roundnode] (9) at (3,-9) {$9$};
            \node[roundnode] (10)at (8,-9) {$10$};
            \node[roundnode] (11)at (2,-12) {$11$};
            \node[roundnode] (12)at (6,-12) {$12$};

            \draw[arrow] (1) to (5);
            \draw[arrow] (1) to[bend left=16] (6);
            \draw[arrow] (1) to (8);
            \draw[arrow] (1) to (9);

            \draw[arrow] (2) to (6);
            \draw[arrow] (2) to (7);
            \draw[arrow] (2) to (8);

            \draw[arrow] (3) to (6);
            \draw[arrow] (3) to (9);
            \draw[arrow] (3) to (11);

            \draw[arrow] (4) to (6);
            \draw[arrow] (4) to (7);
            \draw[arrow] (4) to (8);
            \draw[arrow] (4) to (9);

            \draw[arrow] (5) to (11);

            \draw[arrow] (6) to (9);
            \draw[arrow] (6) to (10);
            \draw[arrow] (6) to[bend left=16] (11);
            \draw[arrow] (6) to (12);

            \draw[arrow] (7) to[bend left=8] (11);
            \draw[arrow] (7) to (12);

            \draw[arrow] (8) to (10);
            \draw[arrow] (8) to[bend left=20] (12);


            \node (3in) at ([yshift=5cm]3) {};
            \draw[arrow, dashed] (3in) to (3);
            
            \node (4in) at ([yshift=5cm]4) {};
            \draw[arrow, dashed] (4in) to (4);

            \node at (5.5,1.5) {\textit{Солнечный свет}};

            \node (1in) at ([xshift=-3cm]1) {};
            \node (2in) at ([xshift=8cm]2) {};
            
            \draw[arrow] (1in) to (1);
            \draw[arrow] (2in) to node[pos=0.25,anchor=south] {\textit{Вносимая органика}} (2);

            \node at ([xshift=3cm]4) {\textit{Продуценты}};
            \node[align=center] at ([xshift=2cm]8) {\textit{Первичные}\\\textit{консументы}};
            \node[align=center] at ([xshift=4cm]10) {\textit{Промежуточный}\\\textit{уровень}};
            \node[align=center] at ([xshift=5cm]12) {\textit{Вторичные}\\\textit{консументы}};


            % \draw [arrow] (1) to[bend right] node[pos=0.9, anchor=south] {$-$} (0);
            % \draw [arrow] (0) to[bend right] node[pos=0.9, anchor=north] {$+$} (1);

            % \draw [arrow] (2) to[bend right] node[pos=0.9, anchor=south] {$-$} (1);
            % \draw [arrow] (1) to[bend right] node[pos=0.9, anchor=north] {$+$} (2);

            % \draw [arrow] (q) to[bend right] node[pos=0.9, anchor=south] {$-$} (q1);
            % \draw [arrow] (q1) to[bend right] node[pos=0.9, anchor=north] {$+$} (q);

            % \node (d_u) at (8, 0.6) {\(\dots\)};
            % \node (d_d) at (8,-0.6) {\(\dots\)};
            
            % \draw [thick] (q1) to[out=150, in=0] (d_u);
            % \draw [arrow] (d_u) to[out=180, in=30] node[pos=0.8, anchor=south] {$-$} (2);
            
            % \draw [thick] (2) to[out=-30, in=180] (d_d);
            % \draw [arrow] (d_d) to[out=0, in=-150] node[pos=0.8, anchor=north] {$+$} (q1);
            
            % \draw[arrow] (0) edge [in=-150, out=150, looseness=6] node [pos=0.9, anchor=north] {$-$} (0);
    
        \end{tikzpicture}
        \caption{Часть трофической сети экосистемы ручья в Южном Уэльсе.} \label{small_river_graph}
    \end{figure}

    Можно сказать, что трофическая цепь описывает сообщество, два последовательных вида которого образуют пару хищник -- жертва. Трофическая цепь начинается с некоторого ресурса.

    Поскольку реальное сообщество описывается достаточно сложной трофической цепью, то и модель усложняется. Есть два пути упрощения исходной модели. Первый это агрегация всех видов, принадлежащих одному и тому же трофическому уровню в один <<псевдовид>>, в случае достаточно близких экологических характеристик видов уровней. Второй это выделение в трофической сети одной вертикальной ветви поток энергии, который намного превосходит потоки энергии по другим ветвям, и пренебрежение остальными, в случае присутствия доминантного вида. В лбом случае после таких упрощений на каждом из уровней останется один вид, а трофическая структура этого сообщества будет описываться трофической цепью. В случае невозможности осреднения или выделения доминантной ветви, необходимо будет рассматривать несколько трофических цепей или ветвящиеся трофические цепи.

    В данной работе содержится исследование устойчивости подобного рода трофических структур в рамках моделей <<почти вольтерровского типа>>.