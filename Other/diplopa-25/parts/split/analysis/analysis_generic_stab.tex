\subsection{Условия существования}
Пусть для ветвящейся трофической цепи, изображённой на рис. \ref{schema_split}, параметры системы таковы, что существует равновесие типа 
\[ N^* = [ N_0, N_1, \dots, N_q; N'_1,\dots, N'_r ], \]
и всегда выполняются условия \( k_{q+1} V_q(N_q) < m_{q+1}, k'_{r+1} V'_r(N'_r) < m'_{r+1} \). Эти условия ограничивают сверху длину обоих цепей.

Для анализа будем использовать систему дифференциальных уравнений в матричной записи, линеаризованную в окрестности равновесного состояния:
\begin{equation} \label{double_matrix_system}
    \begin{split}
        & \frac{d \mb{x}}{dt} = B \mb{x} + b_1 x_0 \mb{\delta_1} - d^b_s x'_1 \mb{\delta_s}, \\
        & \frac{d \mb{x'}}{dt}= B' \mb{x'} + b'_1 x_s \mb{\delta_1}, \\
        & \frac{d x_0}{dt} = \mb{c}^T \mb{x} + \mb{c'}{}^T \mb{x'} - b_0 x_0.
    \end{split}
\end{equation}

Здесь \( \mb{x}_i = N_i - N^*_i; ~ \mb{x'}_k = N'_k - N^{'*}_k \). \( \mb{x}, \mb{c} \) -- \(q\)-мерные вектор-столбцы с компонентами \(\mb{x} = [x_1, x_2, \dots, x_q]^T, ~ \mb{c} = [c_1 - d_0, c_2, \dots, c_q]^T\). \( \mb{x'}, \mb{c'} \) -- \(r\)-мерные вектор-столбцы с компонентами \( \mb{x'} = [x'_1, \dots, x'_r]^T, ~ \mb{c'} = [c'_1, \dots, c'_r]^T \). \( \mb{\delta_j} \) -- вектор-столбец, все компоненты которого равны \(0\), кроме \(j\)-ой, равной \(1\); \(d^b_s = N_s V^b_s(N_s)\), где \( V^b_s(N_s) \) -- трофическая функция \(1\)-го уровня боковой цепи, зависящая от численности \(s\)-го уровня главной: \( b'_1 = k'_1 N'_1 \cdot \frac{d V^b_s}{d N_s} (N_s) \). И матрицы \(B\) и \(B'\) -- якобиевы (трёхдиагональные) матрицы порядков \(q \times q\) и \( r \times r \) типа:
\begin{equation*}
    B = \left( \begin{matrix}
        -h_1 & -d_1 &         &         & \mb{0}    \\
        b_2 & -h_2 & -d_2 \\
            & \ddots & \ddots & \ddots \\
            &        & b_{q-1} & -h_{q-1} & -d_{q-1} \\
        \mb{0}&      &         & b_q     & -h_q
    \end{matrix} \right),
\end{equation*}
при \(b_i, d_i > 0, ~ h_i \geq 0\).

Для определения устойчивости будем искать функцию Ляпунова в виде:
\begin{equation}
    V(\mb{x}, \mb{x'}, x_0) = \mb{x}^T G \mb{x} + \mb{x'}{}^T G' \mb{x'} + \frac{1}{2} x^2_0,
\end{equation}
где \(G, G'\) -- диагональные матрицы с элементами \(g_i, g'_k > 0 ~ (i=\overline{1,q}; ~ l=\overline{1,r})\). Вычислим производную этой функции
\begin{equation*}
    \frac{dV}{dt} = \frac{\D V}{\D \mb{x}} \frac{d \mb{x}}{d t} + \frac{\D V}{\D \mb{x'}} \frac{d \mb{x'}}{d t} + \frac{\D V}{\D x_0} \frac{d x_0}{d t}
\end{equation*}
Частные производные квадратичных форм равны \( \frac{\D V}{\D \mb{x}} = (G + G^T) \mb{x} \), а \(\frac{\D V}{\D x_0} = x_0 \).

Первое слагаемое:
\[ 
\begin{split}
    \frac{\D V}{\D \mb{x}} \frac{d \mb{x}}{d t} &= (G + G^T) \mb{x} \cdot (B \mb{x} + b_1 x_0 \mb{\delta_1} - d^b_s x'_1 \mb{\delta_s}) = \\
    &= \mb{x}^T (G + G^T) B \mb{x} + (G + G^T) \mb{x} \cdot (b_1 x_0 \mb{\delta_1} - d^b_s x'_1 \mb{\delta_s}) = \\
    &= \mb{x}^T (B^T G + G B) \mb{x} + 2 b_1 x_0 (\mb{\delta_1}^T G \mb{x}) - 2 d^b_s x'_1 (\mb{\delta_s}^T G \mb{x})
\end{split}
\]

Второе слагаемое:
\[ 
\begin{split}
    \frac{\D V}{\D \mb{x'}} \frac{d \mb{x'}}{d t} &= (G' + G'{}^T) \mb{x} \cdot (B' \mb{x'} + b'_1 x_s \mb{\delta_1}) = \\
    &= \mb{x'}{}^T (G' + G'{}^T) B' \mb{x'} + (G' + G'^T) \mb{x'} \cdot b'_1 x_s \mb{\delta_1} = \\
    &= \mb{x'}{}^T (B'{}^T G' + G' B') \mb{x'} + 2 b'_1 x_s (\mb{\delta_1}^T G' \mb{x'})
\end{split}
\]

Третье слагаемое:
\[ 
\begin{split}
    \frac{\D V}{\D x_0} \frac{d x_0}{d t} &= x_0 \cdot (\mb{c}^T \mb{x} + \mb{c'}{}^T \mb{x'} - b_0 x_0) = \\
    & = x_0(\mb{c}^T \mb{x} + \mb{c'}{}^T \mb{x'}) - b_0 x_0^2
\end{split}
\]

Сделаем замены:
\begin{equation}
    \begin{split}
        & B^T G + G B = -F, ~~ B'{}^T G' + G' B' = -F', \\
        & \mb{l} = -b_1 g_1 \mb{\delta_1} - \frac{1}{2} \mb{c}, ~~ \mb{l'} = -\frac{1}{2} \mb{c'}.
    \end{split}
\end{equation}

В сумме получаем:
\begin{equation}
    \begin{split}
        \frac{dV}{dt} = \mb{x}^T (-F) \mb{x} &+ \mb{x'}{}^T (-F') \mb{x'} - 2 x_0(\mb{l}^T \mb{x} + \mb{l'}{}^T \mb{x'}) - \\
        &- 2 d^b_s x'_1 (\mb{\delta_s}^T G \mb{x}) + 2 b'_1 x_s (\mb{\delta_1}^T G' \mb{x'}) - b_0 x_0^2.
    \end{split}
\end{equation}

Для определения устойчивости нужна противоположная величина:
\begin{equation} \label{full_diff_split}
    \begin{split}
        -\frac{dV}{dt} = \mb{x}^T F \mb{x} &+ \mb{x'}{}^T F' \mb{x'} + 2 x_0(\mb{l}^T \mb{x} + \mb{l'}{}^T \mb{x'}) + \\
        & + 2 d^b_s x'_1 (\mb{\delta_s}^T G \mb{x}) - 2 b'_1 x_s (\mb{\delta_1}^T G' \mb{x'}) + b_0 x_0^2.
    \end{split}
\end{equation}

Матрица \(F\) и её преставление является матричным соотношением Ляпунова, которое для устойчивой матрицы \(B\) должно удовлетворятся при некоторых положительно определённых матрицах \(G\) и \(F\). Поскольку матрица \(B\) и вид матрицы \(G\) определены, это уравнение имеет единственное решение, которое задаётся элементами матрицы \(F\).
\begin{equation*}
    f_{ij} = \left\{ \begin{split}
        2h_i g_i, \quad & i=j=\overline{1,q}, \\
        d_{i-1} g_{i-1} - b_i g_i, \quad & i=\overline{2,q}, ~~ j=i-1, \\
        d_i g_i - b_{i+1} g_{i+1}, \quad & i=\overline{1,q-1}, ~~ j=i+1, \\
        0, \quad & \abs{i-j} \geq 2.
    \end{split} \right.
\end{equation*}

Пусть \(f_{ij} = f_{ji} = 0, ~~ i \neq j\), тогда
\begin{equation}
    g_i = g_{i-1} \frac{d_{i-1}}{b_i} = \cdots = g_1 \frac{d_1 d_2 \dots d_{i-1}}{b_2 b_3 \dots d_i} = g_1 \widetilde{g}_i, \quad i=\overline{2,q}.
\end{equation}

Если при этом \( b_i > 0, ~ h_i > 0 \), то \(F\) -- диагональная матрица с положительными элементами, что делает её положительно определённой. Аналогичными рассуждениями делаем такие же выводы о матрице \(F'\) с обозначениями:
\begin{equation}
    g'_k = g'_{k-1} \frac{d'_{k-1}}{b'_k} = g'_1 \frac{d'_1 d'_2 \dots d'_{k-1}}{b'_2 b'_3 \dots d'_k} = g'_1 \widetilde{g}'_k, \quad k=\overline{2,r}.
\end{equation}

Выберем \(g'_1\) таким образом, что бы
\begin{equation} \label{split_delta_g1}
    d^b_s x'_1 (\mb{\delta_s}^T G \mb{x}) = b'_1 x_s (\mb{\delta_1}^T G' \mb{x'}) \Rightarrow 
    d^b_s x'_1 g_s x_s = b'_1 x_s g'_1 x'_1,
\end{equation}
откуда,
\begin{equation}
    g'_1 = g_s \frac{d^b_s}{b'_1} = g_s \frac{d_1 d_2 \dots d_{s-1} d^b_s}{b'_1 b_2 \dots b_s} = g_1 \Lambda_s.
\end{equation}

Выделим в выражении (\ref{full_diff_split}) полные квадраты:
\begin{equation*}
    \begin{split}
        & -\frac{dV}{dt} = (\mb{x}^T F \mb{x} + 2 x_0 \mb{l}^T \mb{x}) + (\mb{x'}{}^T F' \mb{x'} + 2 x_0 \mb{l'}{}^T \mb{x'}) + b_0 x_0^2 = \\
        & = \mb{x}^T F \mb{x} + x_0 \mb{l}^T (F^{-1} F) \mb{x} + x_0 \mb{x}^T (F F^{-1}) \mb{l} + \mb{l}^T F^{-1} \mb{l} x^2_0 - \mb{l}^T F^{-1} \mb{l} x^2_0 + \\
        & + \mb{x'}{}^T F' \mb{x'} + x_0 \mb{l'}{}^T (F'{}^{-1} F') \mb{x'} + x_0 \mb{x'}{}^T (F' F'{}^{-1}) \mb{l'} + \\
        & + \mb{l'}{}^T F'{}^{-1} \mb{l'} x^2_0 - \mb{l'}{}^T F'{}^{-1} \mb{l'} x^2_0 + b_0 x_0^2 = \\
        & = (\mb{x}^T + x_0 \mb{l}^T F^{-1})F(\mb{x} + x_0 F^{-1} \mb{l}) + \\
        & + (\mb{x'}{}^T + x_0 \mb{l'}{}^T F'{}^{-1})F'(\mb{x'} + x_0 F{'}^{-1} \mb{l'}) + (b_0 - \mb{l}^T F^{-1} \mb{l} - \mb{l'}{}^T F'{}^{-1} \mb{l'}) x^2_0.
    \end{split}
\end{equation*}

Отсюда, вследствие положительной определённости \(F\) следует достаточное условие для отрицательной определённости \(\frac{dV}{dt}\):
\begin{equation} \label{split_b_dost}
    b_0 > \mb{l}^T F^{-1} \mb{l} + \mb{l'}{}^T F'{}^{-1} \mb{l'} = \sum_{i=1}^{q} \frac{l^2_i}{2h_i g_1 \wt{g}_i} + \sum_{k=1}^{r} \frac{l'{}^2_k}{2h_k g'_1 \wt{g}'_i}.
\end{equation}

При условии существования \(g_1 > 0\) функция \(V(\mb{x}, \mb{x'}, x_0)\) является функцией Ляпунова поскольку она равна \(0\) только в точке \([\mb{0}, \mb{0}, 0]\), а иначе \(V > 0\).

Покажем, что такое \(g_1\) существует. Выпишем конкретный вид \(\mb{l}\):
\begin{equation}
    \begin{split}
        l_1 &= -b_1 g_1 - \frac{1}{2} (c_1 - d_0) = -b_1 g_1 + \frac{1}{2} (d_0 - c_1) = -b_1 g_1 + \frac{1}{2} \wt{d}_0, \\
        l_i &= -\frac{1}{2} c_i, \quad i=\overline{2,q}.
    \end{split}
\end{equation}

Выделим \(l_1\) в неравенстве (\ref{split_b_dost}) и запишем в виде квадратного трёхчлена:
\begin{equation} \label{split_quad_dost}
    \begin{split}
        &b_0 > \frac{g_1^2 b_1^2 - g_1 b_1 \wt{d}_0 + \frac{1}{4} \wt{d}_0^2 }{2h_1 g_1 \wt{g}_1} + \sum_{i=2}^{q} \frac{l^2_i}{2h_i g_1 \wt{g}_i} + \sum_{k=1}^{r} \frac{l'{}^2_k}{2h_k g_1 \Lambda_s \wt{g}'_i}, \\
        &g_1^2 b_1^2 - (2b_0 h_1 + b_1 \wt{d}_0) g_1 + \frac{1}{4} \left(\wt{d}_0^2 + h_1 \Phi \right) < 0,
    \end{split}
\end{equation}
где \( \Phi = \sum_{i=2}^{q} \frac{c_i^2}{h_i \wt{g}_i} + \sum_{k=1}^{r} \frac{c'_i{}^2}{h_i \Lambda_s \wt{g}'_i} \).

Достаточно найти \(\min g_1\), который достигается в вершине параболы, и потребовать чтобы он достигался при \(\min g_1 > 0\).
\begin{equation}
    \min g_1 = \frac{2b_0 h_1 + b_1 \wt{d}_0}{2 b_1^2} > 0,
\end{equation}
и при \(2b_0 h_1 > - b_1 \wt{d}_0\) эта величина положительна. Подставим \(\min g_1\) в (\ref{split_quad_dost}):
\begin{equation*}
    \begin{split}
        & b_1^2 \frac{(2b_0 h_1 + b_1 \wt{d}_0)^2}{4 b_1^4} - \frac{(2b_0 h_1 + b_1 \wt{d}_0)^2}{2 b_1^2} + \frac{\wt{d}_0^2}{4} + \frac{1}{4} h_1 \Phi < 0, \\
        & \frac{(2b_0 h_1 + b_1 \wt{d}_0)^2}{4 b_1^2} - \frac{\wt{d}_0^2}{4} > \frac{1}{4} h_1 \Phi, \\
        & \frac{4 b_0^2 h_1^2 + 4 b_0 b_1 h_1 \wt{d}_0 + \wt{d}_0^2 b_1^2 - \wt{d}_0^2 b_1^2}{4 b_1^2} > \frac{1}{4} h_1 \Phi, \\
        & b_0 h_1 (b_0 h_1 + b_1 \wt{d}_0) > \frac{b_1^2}{4} h_1 \Phi,
    \end{split}
\end{equation*}
и получим достаточное условие в виде:
\begin{equation} \label{split_h_dost}
    b_0 \left[ b_1 (d_0 - c_1) + b_0 h_1 \right] > \frac{b_1^2}{4} \left( \sum_{i=2}^{q} \frac{c_i^2}{h_i \wt{g}_i} + \sum_{k=1}^{r} \frac{c'_i{}^2}{h_i \Lambda_s \wt{g}'_i} \right).
\end{equation}

Если левая часть этого неравенства не зависит от параметров боковой цепи, то его правая часть увеличивается при появлении этой цепи. То есть ветвление приводит к уменьшению области устойчивости и тем самым привод к снижению области устойчивости всей системы. Если же боковая цепь не замкнута, то ветвление не меняет устойчивости цепи. И если степень замыкания мала (\(c_i, c'_k \ll 1\)), то цепь наверняка устойчивая.

Таким образом, при условиях \(h_i, h'_k > 0, ~~ 2 b_0 h_1 > b_1 (c_1 - d_0)\) и (\ref{split_h_dost}) ветвящаяся трофическая цепь устойчива.

Пусть теперь \(c_i, c'_k = 0 ~ \forall i \neq 1, k \), и \(h_i, h'_k = 0, ~ \forall i, k \). Вернёмся к выражению (\ref{full_diff_split}) с условиями \(f_{ij} = f_{ji} = 0, ~ i \neq j\) и (\ref{split_delta_g1}):
\begin{equation}
    - \frac{dV}{dt} = \left( \frac{1}{2} \wt{d}_0 - b_1 g_1 \right) x_0 x_1 + b_0 x_0^2.
\end{equation}
Выбирая \(g_1 = \frac{\wt{d}_0}{2 b_1} > 0\) получаем \(\frac{dV}{dt} \leq 0\). То есть при \(\wt{d}_0 = d_0 - c_1 > 0\) имеем простую устойчивость. Однако, можем воспользоваться теоремой Барбашина-Красовского\cite{barabashin_stability} для определения асимптотической устойчивости.

\begin{theorem}[Барбашина-Красовского] \label{theorem_bar_kras}
    Пусть в \(U_\veps(0)\) (\(\veps\)-окрестности равновесия \(\mb{x} = 0\)) существует функция \(V(\mb{x}) \in C^1\left( U_\veps(0) \right)\) такая, что
    \begin{enumerate}
        \item \(V(\mb{x}) > 0, ~ \mb{x} \in U_\veps(0) \backslash \{\mb{0}\}\),
        \item \(V(\mb{0}) = 0\),
        \item \(\frac{dV}{dt}(\mb{x}) \leq 0, ~ \mb{x} \in U_\veps(0) \), причём подмножество фазового пространства \( M = \left\{ \mb{x} ~ \Big\vert ~ \frac{dV}{dt}(\mb{x}) = 0 \right\} \) не содержит целых решений системы, отличных от \(\mb{x} = 0\).
    \end{enumerate}
    Тогда это положение равновесия асимптотически устойчиво.
\end{theorem}

Здесь подмножество равно \( M = \{ x_0 = 0 \} \). Выясним, есть ли в \(M\) решения, отличные от \( [x_0 = 0, ~ \mb{x} = \mb{0}, ~ \mb{x'} = \mb{0}] \). Если \( \mb{x}, \mb{x'} \neq \mb{0} \), то \(\frac{dx_0}{dt} = -(d_0 - c_1) x_1\), что значит, любое решение выйдет из \(M\). При \(x_0 = x_1 = 0\), а остальные не равны нулю, получаем \(\frac{dx_1}{dt} = -d_1 x_2\), что аналогично выведет решение во вне. И так аналогично до \( x_1 = x_2 = \dots = x_{s-1} = 0 \). При \(x_1 = \dots = x_{s} = 0\) получим \( \frac{d x_s}{dt} = -d_{s} x_{s+1} - d^b_s x'_1 \), значит выводит при \( \abs{x_s} + \abs{x'_1} > 0\). Далее аналогично по цепочке идёт в обоих ветвях до последних двух уравнений. которые будут в виде \(\frac{dx_{q-1}}{dt} = -d_{q-1} x_q, ~ \frac{d x_q}{dt} = b_q x_{q-1}\). При \(x_{q-1} = 0 ~ x_q = \const \), то если константа не равна нулю, решение уходит из \(M\). Следовательно, от предположения противоположного получили, что в \(M\) только единственное целое решение, равное \( [x_0 = 0, ~ \mb{x} = \mb{0}, ~ \mb{x'} = \mb{0}] \). Таким образом, условия теоремы \eqref{theorem_bar_kras} выполнены, значит это положение равновесия асимптотически устойчивое.

В итоге, при условиях \(c_i, c'_k = 0 ~ \forall i \neq 1, k\) и \( h_i, h'_k = 0, ~ d_0 > c_1\) ветвящаяся трофическая цепь асимптотически устойчива.
