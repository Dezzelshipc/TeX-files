\subsection{Численные эксперименты}
Чтобы рассмотреть разные варианты ветвящейся цепи, выберем максимальные длины ветвей, равные \(q = 4, ~ r=2\). Таким образом, рассматривая четыре эксперимента по две пары параметров: \(s=2, ~~ s=3\) и \(\alpha^b_s = 16, ~~ \alpha^b_s = 8\). Пусть линейная система \eqref{double_lin} имеет такие коэффициенты:

\begin{table}[H]
    \centering
    \caption{Коэффициенты дифференциальной системы ветвящейся цепи}
    \begin{tabular}{|c|c|c|c|c|c|}
        \hline
        \(i\)        & \(0\) & \(1\) & \(2\) & \(3\) & \(4\) \\ \hline
        \(\alpha_i\)& \(20\) & \(17.5\) & \(15\) & \(12.5\) & \(10\) \\ \hline
        \(m_i\) &            & \(5\) & \(4\) & \(3\) & \(2\) \\ \hline
        \(k_i\) &            & \(0.5\) & \(0.4\) & \(0.3\) & \(0.2\) \\ \hline
        \(a_i\) &            & \(0.2\) &  &  &  \\ \hline \hline

        \(\alpha'_i\)& \(\alpha^b_s\) & \(16\) & \(8\) &  &  \\ \hline
        \(m'_i\) &            & \(4\) & \(1\) &  &  \\ \hline
        \(k'_i\) &            & \(0.5\) & \(0.3\) &  &  \\ \hline
    \end{tabular}
\end{table}

Варьируем значение \(Q\) и получаем графики численностей в равновесии. Используем метод Рунге-Кутты \(4\)-порядка с шагом \(h = 0.01\). Начальные значения численностей равны \(0.5\). 

Рассмотрим при \(s=2\) и \(\alpha^b_s = 16\). Поскольку меняется длина сразу двух ветвей, то получим таблицу состояний:
\begin{table}[H]
    \centering
    \caption{Таблица состояний при \(s=2\) и \(\alpha^b_s = 16\).}
    \begin{tabular}{|c|c|c|c|}
        \hline
        \backslashbox{\(q\)}{\(r\)} & \(0\) & \(1\) & \(2\) \\ \hline
        \(1\) 
        & \cellcolor{gray!20}\(\begin{matrix} q(0,5.14) \\ r?(-0.87, -0.36) \end{matrix}\) 
        & -- 
        & \(\begin{matrix} q(4.28, 9.42) \\ r?(-0.87, -0.87) \end{matrix}\) \\ \hline
        \(2\) 
        & \cellcolor{gray!20}\(\begin{matrix} q(5.14,18.47) \\ r(9.3, 15.14) \end{matrix}\) 
        & \cellcolor{gray!20}\(\begin{matrix} q?(-0.87,0.29) \\ r(15.14, 27.76) \end{matrix}\) 
        & \cellcolor{gray!20}\(\begin{matrix} q(9.42, 33.87) \\ r(27.76, +\infty) \end{matrix}\) \\ \hline
        \(3\) 
        & \(\begin{matrix} q(18.47,73.9) \\ r?(0.29,0.8) \end{matrix}\) 
        & -- 
        & \cellcolor{gray!20}\(\begin{matrix} q(33.87, 89.3) \\ r?(0.29, 0.29) \end{matrix}\) \\ \hline
        \(4\) 
        & \(\begin{matrix} q(73.9,+\infty) \\ r(37.23. 60.57) \end{matrix}\) 
        & \(\begin{matrix} q?(0.29, 0.29) \\ r(60.57, 73.19) \end{matrix}\) 
        & \cellcolor{gray!20}\(\begin{matrix} q(89.3, +\infty) \\ r?(73.19, +\infty) \end{matrix}\) \\ \hline
    \end{tabular}
\end{table}
Здесь обозначениями \(q(x,y), ~ r(x,y)\) показаны нижняя и верхняя границы \((x, y)\) поступающей энергии \(Q\) для данных длин цепи (\(x < Q < y\)), а \(q?(x,y), ~ r?(x,y)\) обозначены левая и правая часть неравенства существования соответствующих ветвей (\(x < 0 < y\)). Серым цветом помечены ячейки, длины которых появляются в эксперименте.

\begin{figure}[H]
    \centering
    \subfigexp{1}   {pictures/split/exp1_s2_16/Q}{.3}
    \subfigexp{10}  {pictures/split/exp1_s2_16/Q}{.3}
    \subfigexp{20}  {pictures/split/exp1_s2_16/Q}{.3}
    \subfigexp{30}  {pictures/split/exp1_s2_16/Q}{.3}
    \subfigexp{40}  {pictures/split/exp1_s2_16/Q}{.3}
    % \subfigexp{50}  {pictures/split/exp1_s2_16/Q}{.3}
    \subfigexp{60}  {pictures/split/exp1_s2_16/Q}{.3}
    % \subfigexp{80}  {pictures/split/exp1_s2_16/Q}{.3}
    \subfigexp{90}  {pictures/split/exp1_s2_16/Q}{.3}
    % \subfigexp{100} {pictures/split/exp1_s2_16/Q}{.3}
    \subfigexp{1000}{pictures/split/exp1_s2_16/Q}{.3}
\caption{Численности видов системы при \(s=2\) и \(\alpha^b_s = 16\).}  \label{fig:split_exp1_s2_16}
\end{figure}

Рассмотрим при \(s=2\) и \(\alpha^b_s = 8\). 
\begin{table}[H]
    \centering
    \caption{Таблица состояний при \(s=2\) и \(\alpha^b_s = 8\).}
    \begin{tabular}{|c|c|c|c|}
        \hline
        \backslashbox{\(q\)}{\(r\)} & \(0\) & \(1\) & \(2\) \\ \hline
        \(1\)
        & \(\begin{matrix} q(0.0, 5.14) \\ r?(-1.75, -0.36) \end{matrix}\) \cellcolor{gray!20}
        & --
        & \(\begin{matrix} q(2.14, 7.28) \\ r?(-1.75, -1.75) \end{matrix}\) \\ \hline
        \(2\)
        & \(\begin{matrix} q(5.14, 18.47) \\ r(9.3, 25.14) \end{matrix}\) \cellcolor{gray!20}
        & \(\begin{matrix} q?(-1.75, -0.58) \\ r(25.14, 35.61) \end{matrix}\)
        & \(\begin{matrix} q(7.28, 26.17) \\ r(35.61, +\infty) \end{matrix}\) \\ \hline
        \(3\)
        & \(\begin{matrix} q(18.47, 73.9) \\ r?(-0.58, 0.8) \end{matrix}\) \cellcolor{gray!20}
        & --
        & \(\begin{matrix} q(26.17, 81.6) \\ r?(-0.58, -0.58) \end{matrix}\) \\ \hline
        \(4\)
        & \(\begin{matrix} q(73.9, +\infty) \\ r(37.23, 100.57) \end{matrix}\) \cellcolor{gray!20}
        & \(\begin{matrix} q?(-0.58, -0.58) \\ r(100.57, 111.04) \end{matrix}\) \cellcolor{gray!20}
        & \(\begin{matrix} q(81.6, +\infty) \\ r(111.04, +\infty) \end{matrix}\) \cellcolor{gray!20} \\ \hline
    \end{tabular}
\end{table}
\begin{figure}[H]
    \centering
    \subfigexp{1}   {pictures/split/exp1_s2_8/Q}{.3}
    \subfigexp{10}  {pictures/split/exp1_s2_8/Q}{.3}
    \subfigexp{30}  {pictures/split/exp1_s2_8/Q}{.3}
    \subfigexp{70}  {pictures/split/exp1_s2_8/Q}{.3}
    \subfigexp{90}  {pictures/split/exp1_s2_8/Q}{.3}
    \subfigexp{105} {pictures/split/exp1_s2_8/Q}{.3}
    \subfigexp{120} {pictures/split/exp1_s2_8/Q}{.3}
    \subfigexp{1000}{pictures/split/exp1_s2_8/Q}{.3}
\caption{Численности видов системы при \(s=2\) и \(\alpha^b_s = 8\).}  \label{fig:split_exp1_s2_8}
\end{figure}


Рассмотрим при \(s=3\) и \(\alpha^b_s = 16\). 
\begin{table}[H]
    \centering
    \caption{Таблица состояний при \(s=3\) и \(\alpha^b_s = 16\).}
    \begin{tabular}{|c|c|c|c|}
        \hline
        \backslashbox{\(q\)}{\(r\)} & \(0\) & \(1\) & \(2\) \\ \hline
        \(1\) 
        & \cellcolor{gray!20}\(\begin{matrix} q(0,5.14) \\ r(9.16, 14.78) \end{matrix}\) 
        & \(\begin{matrix} q?(-1.64, -1.07) \\ r(14.78, 57.37) \end{matrix}\)
        & \(\begin{matrix} q(0, 19.95) \\ r(57.37, +\infty) \end{matrix}\) \\ \hline
        \(2\) 
        & \cellcolor{gray!20}\(\begin{matrix} q(5.14,18.47) \\ r?(-1.07, -0.44) \end{matrix}\) 
        & -- 
        & \(\begin{matrix} q(19.95, 33.29) \\ r?(-1.07, -1.07) \end{matrix}\) \\ \hline
        \(3\) 
        & \cellcolor{gray!20}\(\begin{matrix} q(18.47,73.9) \\ r(32.91, 53.11) \end{matrix}\) 
        & \cellcolor{gray!20}\(\begin{matrix} q?(-1.07, 0.64) \\ r(53.11, 95.11) \end{matrix}\) 
        & \cellcolor{gray!20}\(\begin{matrix} q(33.29, 133.16) \\ r(95.71, +\infty) \end{matrix}\) \\ \hline
        \(4\) 
        & \(\begin{matrix} q(73.9,+\infty) \\ r?(0.64, 1.26) \end{matrix}\) 
        & -- 
        & \cellcolor{gray!20}\(\begin{matrix} q(133.16, +\infty) \\ r?(0.64, 0.64) \end{matrix}\) \\ \hline
    \end{tabular}
\end{table}
\begin{figure}[H]
    \centering
    \subfigexp{1}   {pictures/split/exp1_s3_16/Q}{.3}
    \subfigexp{10}  {pictures/split/exp1_s3_16/Q}{.3}
    \subfigexp{20}  {pictures/split/exp1_s3_16/Q}{.3}
    \subfigexp{30}  {pictures/split/exp1_s3_16/Q}{.3}
    \subfigexp{40}  {pictures/split/exp1_s3_16/Q}{.3}
    \subfigexp{80}  {pictures/split/exp1_s3_16/Q}{.3}
    \subfigexp{120} {pictures/split/exp1_s3_16/Q}{.3}
    \subfigexp{150} {pictures/split/exp1_s3_16/Q}{.3}
    \subfigexp{1000}{pictures/split/exp1_s3_16/Q}{.3}
\caption{Численности видов системы при \(s=3\) и \(\alpha^b_s = 16\).}  \label{fig:split_exp1_s3_16}
\end{figure}


Рассмотрим при \(s=3\) и \(\alpha^b_s = 8\). 
\begin{table}[H]
    \centering
    \caption{Таблица состояний при \(s=3\) и \(\alpha^b_s = 8\).}
    \begin{tabular}{|c|c|c|c|}
        \hline
        \backslashbox{\(q\)}{\(r\)} & \(0\) & \(1\) & \(2\) \\ \hline
        \(1\)
        & \(\begin{matrix} q(0.0, 5.14) \\ r(9.16, 24.42) \end{matrix}\) \cellcolor{gray!20}
        & \(\begin{matrix} q?(-2.71, -2.14) \\ r(24.42, 59.61) \end{matrix}\)
        & \(\begin{matrix} q(0.0, 12.55) \\ r(59.61, +\infty) \end{matrix}\) \\ \hline
        \(2\)
        & \(\begin{matrix} q(5.14, 18.47) \\ r?(-2.14, -0.44) \end{matrix}\) \cellcolor{gray!20}
        & --
        & \(\begin{matrix} q(12.55, 25.88) \\ r?(-2.14, -2.14) \end{matrix}\) \\ \hline
        \(3\)
        & \(\begin{matrix} q(18.47, 73.9) \\ r(32.91, 87.76) \end{matrix}\) \cellcolor{gray!20}
        & \(\begin{matrix} q?(-2.14, -0.42) \\ r(87.76, 122.94) \end{matrix}\)
        & \(\begin{matrix} q(25.88, 103.53) \\ r(122.94, +\infty) \end{matrix}\) \\ \hline
        \(4\)
        & \(\begin{matrix} q(73.9, +\infty) \\ r?(-0.42, 1.26) \end{matrix}\) \cellcolor{gray!20}
        & --
        & \(\begin{matrix} q(103.53, +\infty) \\ r?(-0.42, -0.42) \end{matrix}\) \\ \hline
    \end{tabular}
\end{table}
\begin{figure}[H]
    \centering
    \subfigexp{1}   {pictures/split/exp1_s3_8/Q}{.3}
    \subfigexp{10}  {pictures/split/exp1_s3_8/Q}{.3}
    \subfigexp{20}  {pictures/split/exp1_s3_8/Q}{.3}
    \subfigexp{60}  {pictures/split/exp1_s3_8/Q}{.3}
    \subfigexp{80}  {pictures/split/exp1_s3_8/Q}{.3}
    \subfigexp{100} {pictures/split/exp1_s3_8/Q}{.3}
    \subfigexp{1000}{pictures/split/exp1_s3_8/Q}{.3}
    \subfigexp{1001}{pictures/split/exp1_s3_8/Q}{.3}
    \subfigexp{10000}{pictures/split/exp1_s3_8/Q}{.3}
\caption{Численности видов системы при \(s=3\) и \(\alpha^b_s = 8\).}  \label{fig:split_exp1_s3_8}
\end{figure}

Также рассмотрим пример при \(s=1\) и \(\alpha^b_s = 8\). Этот пример показывает, что цепь может увеличиваться по частям (в отличие от других примеров, где сначала увеличивалась до конца одна из ветвей). 
\begin{table}[H]
    \centering
    \caption{Таблица состояний при \(s=1\) и \(\alpha^b_s = 8\).}
    \begin{tabular}{|c|c|c|c|}
        \hline
        \backslashbox{\(q\)}{\(r\)} & \(0\) & \(1\) & \(2\) \\ \hline
        \(1\)
        & \(\begin{matrix} q(0.0, 5.14) \\ r(1.87, 9.0) \end{matrix}\) \cellcolor{gray!20}
        & \(\begin{matrix} q?(-1.0, -0.42) \\ r(9.0, 12.33) \end{matrix}\)
        & \(\begin{matrix} q(0.0, 7.04) \\ r(12.33, +\infty) \end{matrix}\) \\ \hline
        \(2\)
        & \(\begin{matrix} q(5.14, 18.47) \\ r?(-0.42, 0.36) \end{matrix}\) \cellcolor{gray!20}
        & --
        & \(\begin{matrix} q(7.04, 20.38) \\ r?(-0.42, -0.42) \end{matrix}\) \\ \hline
        \(3\)
        & \(\begin{matrix} q(18.47, 73.9) \\ r(6.73, 32.33) \end{matrix}\) \cellcolor{gray!20}
        & \(\begin{matrix} q?(-0.42, 1.28) \\ r(32.33, 35.66) \end{matrix}\) \cellcolor{gray!20}
        & \(\begin{matrix} q(20.38, 81.52) \\ r(35.66, +\infty) \end{matrix}\) \cellcolor{gray!20} \\ \hline
        \(4\)
        & \(\begin{matrix} q(73.9, +\infty) \\ r?(1.28, 2.07) \end{matrix}\)
        & --
        & \(\begin{matrix} q(81.52, +\infty) \\ r?(1.28, 1.28) \end{matrix}\) \cellcolor{gray!20} \\ \hline
    \end{tabular}
\end{table}
\begin{figure}[H]
    \centering
    \subfigexp{1}   {pictures/split/exp1_s1_8/Q}{.3}
    \subfigexp{10}  {pictures/split/exp1_s1_8/Q}{.3}
    \subfigexp{20}  {pictures/split/exp1_s1_8/Q}{.3}
    \subfigexp{30}  {pictures/split/exp1_s1_8/Q}{.3}
    \subfigexp{34}  {pictures/split/exp1_s1_8/Q}{.3}
    \subfigexp{40}  {pictures/split/exp1_s1_8/Q}{.3}
    \subfigexp{80}  {pictures/split/exp1_s1_8/Q}{.3}
    \subfigexp{100} {pictures/split/exp1_s1_8/Q}{.3}
    \subfigexp{1000}{pictures/split/exp1_s1_8/Q}{.3}
\caption{Численности видов системы при \(s=1\) и \(\alpha^b_s = 8\).}  \label{fig:split_exp1_s1_8}
\end{figure}