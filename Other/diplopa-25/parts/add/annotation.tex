\thispagestyle{empty}
\begin{center}
\refstepcounter{subsection}
\addcontentsline{toc}{section}{АННОТАЦИЯ}
\section*{\center{АННОТАЦИЯ}}
\end{center}
\vspace{-10pt}

В ходе выполнения выпускной квалификационной работы было проведено исследование динамики и устойчивости биологических сообществ, организованных по принципу трофических цепей. На основе теоретического анализа были проведены вычислительные эксперименты для проверки полученных результатов.
В результате исследования удалось получить аналитические выражения, описывающие математические модели трофических цепей. Проведённые расчёты подтвердили их согласованность с экспериментами. 
Полученные формулы позволяют прогнозировать изменения в структуре и динамике трофических цепей при изменении параметров, в частности количества доступной питательной энергии.

In the course of the final qualification work the dynamics and stability of\\ biological communities organized according to the principle of trophic chains a study was conducted. On the basis of theoretical analysis, computational experiments were conducted to verify the obtained results.
As a result of the study, it was possible to obtain analytical expressions describing the mathematical models of trophic chains. The conducted calculations confirmed their consistency with the experiments. 
The obtained formulas allow predicting changes in the structure and dynamics of trophic chains under changing parameters, in particular, the amount of available nutrient energy.