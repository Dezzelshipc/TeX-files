\renewcommand{\refname}{\centering СПИСОК ЛИТЕРАТУРЫ}
\refstepcounter{section}
\addcontentsline{toc}{section}{СПИСОК ЛИТЕРАТУРЫ}
\begin{thebibliography}{100}
    
    \bibitem{svilog}
        Свирежев, Ю. М. Устойчивость биологических сообществ. // Ю. М. Свирежев, Д. О. Логофет -- М.: Наука, 1978.

    \bibitem{barabashin_stability}
        Барбашин Е. А. Введение в теорию устойчивости. -- М.: Наука, 1967, с. 46—47

    \bibitem{quirk_rupert}
        Quirk J. P., Rupert. R Qualitative Economics and the Stability of Equilibrium. // Rev. Econ. Studies, 1965, 32, №92, p.311-326

    \bibitem{jones_river}
        Jones J. R. E. A further ecological study of a calcareous stream in the «Black Mountain» district of South Wales. // J. Anim. Ecol., 1949, 18, № 2, p. 142—159.

    \bibitem{eman}
        Эман Т. И. О некоторых математических моделях биогеоценозов. // Проблемы кибернетики. Вып. 16. М., 1966.

    \bibitem{alekseev_karishev}
        Алексеев А.А, Карышев И.И. Кинетические уравнения для описания динамики биоценозов.
        
    \bibitem{ee_gir} % Актуальность
        Гиричева Е. Е. Сосуществование популяций в модели трофической цепи с учетом всеядности хищника и внутривидовой конкуренции жертв. -- Математическая биология и биоинформатика, 2021, том 16, выпуск 2, с. 394--410


\end{thebibliography}
