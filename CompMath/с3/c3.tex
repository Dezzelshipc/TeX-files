\documentclass[14pt, a4paper, titlepage, fleqn]{extarticle}

\usepackage[russian]{babel}

\usepackage{amsmath}
\usepackage{amssymb}
\usepackage{graphicx}
\usepackage{float}
\usepackage{svg}

\newcommand{\rnc}[1]
    {\MakeUppercase{\romannumeral #1}}
\newcommand{\otv}{\textit{Ответ:} }

\DeclareMathOperator{\D}{\partial}
\DeclareMathOperator{\Ln}{Ln}
\DeclareMathOperator{\Imz}{Im}
\DeclareMathOperator{\Rez}{Re}

\newcommand{\triple}[3]{(x-x_#2)(x-x_#3)+(x-x_#1)(x-x_#3)+(x-x_#1)(x-x_#2)}


\title{Контрольная работа по численному интегрированию}
\author{Держапольский Юрий Витальевич \\ Группа Б9121-01.03.02сп}
\date{}

\begin{document}

    \maketitle

    \section*{Задание}

    % Для функции, заданной таблично, вычислить значение

    % определенного интеграла методом трапеций, сделать

    % уточнение результата экстраполяцией Ричардсона.

    % Сравнить уточненный результат с вычислениями по методу

    Симпсона.

\end{document}