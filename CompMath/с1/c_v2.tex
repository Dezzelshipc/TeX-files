\documentclass[14pt, a4paper, titlepage, fleqn]{extarticle}

\usepackage[russian]{babel}

\usepackage{amsmath}
\usepackage{amssymb}

\begin{document}

    \textit{\underline{Линейное интерполирование в двумерном случае.}}

    Рассмотрим полином \( z = P_1(x,y) = A+Bx+Cy\), который проходит через 
    3 точки \((x_0, y_0, z_0), (x_1, y_1, z_1), (x_2, y_2, z_2)\). 
    
    Найти полином \(P_1(x,y) = A+Bx+Cy\), 
    который проходит через точки \((1,2,5), (3,2,7), (1,2,0)\).
    \\

    Составим систему для нахождения коэффициентов.
    \[
        \left\{
            \begin{split}
                &A + B + 2C = 5 \\
                &A + 3B+ 2C = 7 \\
                &A + B + 2C = 0
            \end{split}
        \right.
    \]
    
    Заметим, что два равенства содержит одинаковые выражения
    в левой части, но разные в правой, значит эта система
    не имеет решений.

    В данном виде нельзя найти полином. Но можно заметить, что координата
    \( y \) у всех точек равна 2. Значит уравнение плоскости, которая проходит
    через эти 3 точки: \( y = 2 \).

\end{document}