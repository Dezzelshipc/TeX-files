\section{Решение теоретических задач}
    \subsection{Задание 1}
        \subsubsection{Постановка задачи}
            Найдите соотношение эквивалентности, связывающее норму\\ $M(A) = n \max_{1 \leq i,j \leq n}{|a_{ij}|}$ с $ || A ||_\infty $. Проверьте экспериментально.
    
        \subsubsection{Решение}
            \noindent
            \[
                ||Ax||_\infty = \max_i \left| \sum_{j} a_{ij} x_j \right| \leq \max_{i,j}| a_{ij} | \sum_{j} |x_j| = \max_{i,j} |a_{ij}| \cdot ||x||_1
            \]
            Отсюда получаем: $ \max_{i,j} |a_{ij}| \geq \frac{||Ax||_\infty}{||x||_1} $. Равенство достигается, когда все элементы матрицы одинаковые. Имеем: $ \max_{i,j} |a_{ij}| = \sup_{x \neq 0} \frac{||Ax||_\infty}{||x||_1} $.

            Далее будем использовать неравенство: $ ||x||_\infty \leq ||x||_1 \leq n ||x||_\infty $. Получим оценку снизу:
            \[
                M(A) = n \max_{i,j}{|a_{ij}|} = n \sup_{x \neq 0} \frac{||Ax||_\infty}{||x||_1} \geq \sup_{x \neq 0} \frac{n||Ax||_\infty}{n||x||_\infty} = ||A||_\infty.
            \]
            Теперь получим оценку сверху:
            \[
                M(A) = n \max_{i,j}{|a_{ij}|} = n \sup_{x \neq 0} \frac{||Ax||_\infty}{||x||_1} \leq n \sup_{x \neq 0} \frac{||Ax||_\infty}{||x||_\infty} = n||A||_\infty.
            \]
            Таким образом, получили следующее соотношение эквивалентности:
            $$ ||A||_\infty \leq M(A) \leq n||A||_\infty $$ 

            Проверим его экспериментально и убедимся в выполнении неравенства:
            \[
                A = \left(
                    \begin{matrix}
                        1 & 10 & 8 \\
                        -5 & 7 & -3 \\
                        0 & -6 & 12 \\
                    \end{matrix}
                \right), \quad
                \begin{matrix}
                    n = 3, \quad
                    ||A||_\infty = 19, \quad
                    M(A) = 36 \\

                    19 \leq 36 \leq 57
                \end{matrix}
            \]


    
    \subsection{Задание 2}
        \subsubsection{Постановка задачи}
            Докажите теоретически и проверьте экспериментально, что число обусловленности $ \mu(A) = \mu(\alpha A) $, где число $ \alpha \neq 0 $.
        
        \subsubsection{Решение}
            Для доказательства по формуле распишем число обусловленности.
            \[
                \mu(\alpha A) = ||\alpha A|| \cdot || (\alpha A)^{-1} || = |\alpha| \cdot ||A|| \cdot |\alpha^{-1}| \cdot || A^{-1} || = 1 \cdot ||A|| \cdot ||A^{-1}|| = \mu(A) 
            \]
            Равенство доказано. 
            
            Проверим его экспериментально, используя число $ \alpha = 2 $, и используя формулу $ \mu(A) = ||A|| \cdot ||A^{-1}|| $:
            \[
                A = \left(
                    \begin{matrix}
                        3 & 4 \\
                        1 & 2
                    \end{matrix}
                \right), \quad
                A^{-1} = \left(
                    \begin{matrix}
                        1 & -2 \\
                        -\frac{1}{2} & \frac{3}{2}
                    \end{matrix}
                \right),
            \]
            \[
                ||A||_1 = 7, \quad
                ||A^{-1}||_1 = 3, \quad
                \mu(A) = 21
            \]

            \[
                \alpha A = \left(
                    \begin{matrix}
                        6 & 8 \\
                        2 & 4
                    \end{matrix}
                \right), \quad
                (\alpha A)^{-1} = \left(
                    \begin{matrix}
                        \frac{1}{2} & -1 \\
                        -\frac{1}{4} & \frac{3}{4}
                    \end{matrix}
                \right),
            \]
            \[
                ||\alpha A||_1 = 14, \quad
                ||(\alpha A^{-1})||_1 = \frac{3}{2}, \quad
                \mu(\alpha A) = 21
            \]
            Получили: $ \mu(A) = \mu(\alpha A) = 21 $.
      
            