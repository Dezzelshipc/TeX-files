\section{Решение теоретических задач}
    \subsection{Задание 1}
        \subsubsection{Постановка задачи}
            Найдите соотношение эквивалентности, связывающее норму\\ $M(A) = n \max_{1 \leq i,j \leq n}{|a_{ij}|}$ с $ || A ||_\infty $. Проверьте экспериментально.
    
        \subsubsection{Решение}
            \[
                ||Ax||_\infty = \max_i \left| \sum_{j} a_{ij} x_j \right| \leq \max_{i,j}| a_{ij} | \sum_{j} |x_j| = \max_{i,j} |a_{ij}| \cdot ||x||_1
            \]
            Отсюда получаем: $ \max_{i,j} |a_{ij}| \geq \frac{||Ax||_\infty}{||x||_1} $. Равенство достигается, когда все элементы матрицы одинаковые. Имеем: $ \max_{i,j} |a_{ij}| = \sup_{x \neq 0} \frac{||Ax||_\infty}{||x||_1} $.

            Далее будем использовать неравенство: $ ||x||_\infty \leq ||x||_1 \leq n ||x||_\infty $. Получим оценку снизу:
            \[
                M(A) = n \max_{i,j}{|a_{ij}|} = n \sup_{x \neq 0} \frac{||Ax||_\infty}{||x||_1} \geq \sup_{x \neq 0} \frac{n||Ax||_\infty}{n||x||_\infty} = ||A||_\infty.
            \]
            Теперь получим оценку сверху:
            \[
                M(A) = n \max_{i,j}{|a_{ij}|} = n \sup_{x \neq 0} \frac{||Ax||_\infty}{||x||_1} \leq n \sup_{x \neq 0} \frac{||Ax||_\infty}{||x||_\infty} = n||A||_\infty.
            \]
            Таким образом, получили следующее соотношение эквивалентности:
            $$ ||A||_\infty \leq M(A) \leq n||A||_\infty $$ 

            Проверим его экспериментально:


    
    \subsection{Задание 2}
        \subsubsection{Постановка задачи}
            Докажите теоретически и проверьте экспериментально, что число обусловленности $ \mu(A) = \mu(\alpha A) $, где число $ \alpha \neq 0 $.
        
        \subsubsection{Решение}
            Для доказательства по определению распишем число обусловленности.
            \[
                \mu(\alpha A) = ||\alpha A|| \cdot || (\alpha A)^{-1} || = |\alpha| \cdot ||A|| \cdot |\alpha^{-1}| \cdot || A^{-1} || = 1 \cdot ||A|| \cdot ||A^{-1}|| = \mu(A) 
            \]
            Равенство доказано. 
            
            Проверим его экспериментально. Для этого используется код в листинге nn.
    
            \noindent
            \begin{tabular}{| l | l | l | l | l |}
                \hline
                № & Размер матриц & Кол-во матриц & $ \alpha $ & $\log_{10}$ макс. разности \\ \hline
                1 & 5 & 10000 & 4 & $ -\infty $ \\ \hline
                2 & 5 & 10000 & 10 & -6 \\ \hline
                3 & 5 & 100000 & Rand(0.1, 100) & -5 \\ \hline
                4 & 10 & 10000 & 4 & $ -\infty $ \\ \hline
                5 & 10 & 10000 & 10 & -6 \\ \hline
                6 & 100 & 1000 & 4 & $ -\infty $ \\ \hline
                7 & 100 & 1000 & 10 & -5 \\ \hline
                  
            \end{tabular}    
            