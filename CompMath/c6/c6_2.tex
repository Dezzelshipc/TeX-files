\documentclass[14pt, a4paper, titlepage, fleqn]{extarticle}

\usepackage{style/style}

\everymath{\displaystyle}

\title{Контрольная работа по решению уравнений и систем уравнений}
\author{Держапольский Юрий Витальевич \\ Группа Б9121-01.03.02сп}
\date{}

\begin{document}

    \maketitle

    \section*{Задание 2}
        Для нахождения положительного корня нелинейного уравнения \\ \( x^6 -5x - 2 = 0 \)  предложен метод простой итерации. Исследовать этот метод и сделать выводы о целесообразности его использования:
        \[ x_{n+1} = \sqrt[6]{5x_n+2} \]

    \section*{Решение}
        \( \varphi(x) = \sqrt[6]{5x+2} \)

        Положительный корень данного уравнения \( x^* \approx 1.4487\dots \)

        Для исследования метода найдём производную: \( \varphi'(x) = \frac{5}{6 (5x + 2)^{\frac{5}{6}}} \).

        \( | \varphi'(x^*) | \approx 0.1306\ldots < 1 \). Значит по теореме достаточного условия сходимости данный метод сходится к корню и его можно использовать для нахождения положительного корня. 



\end{document}