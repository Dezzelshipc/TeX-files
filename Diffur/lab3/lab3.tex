\documentclass[14pt, a4paper, titlepage, fleqn]{extarticle}

\usepackage{style}
\usepackage{titlepage}

\everymath{\displaystyle}

\begin{document}

    \fefutitlepage{ОТЧЁТ}{к лабораторной работе №3 по дисциплине\\ <<Дифференциальные уравнения>>}
    {01.03.02 <<Прикладная математика и информатика>>}{Б9121-01.03.02сп(1)}{Держапольский Ю.В.}
    
    \tableofcontents

    \pagebreak

    \section{Введение}
        В этой лабораторной работе мы будем решать дифференциальные уравнения, неразрешённые относительно производной, и уравнения высших порядков, находить значение функции и строить её график с помощью производной.

    \pagebreak

    \section{Задание 1}
        \subsection{Постановка задачи}
            Для следующих дифференциальных уравнений указать вид, дать характеристику и найти общее решение с помощью программ компьютерной математики:

            \begin{enumerate}
                \item \( (r-r') \ln{r} = r' \left( \varphi - \ln{r'} \right); \)
                \item \( \tan{\frac{r}{r'}} = \ln{r}; \)
                \item \( r = \frac{3}{2} \varphi r' + e^{r'}; \)
                \item \( \dot{x}^2 - 2x \dot{x} = x^2 \cdot \left( e^{2t} - 1 \right); \)
                \item \( \ln{\theta} = \ln{r'} + r'^2 - 1; \)
            \end{enumerate}

        
        \subsection{Решение}
            \begin{enumerate}
                \item \( (r-r') \ln{r} = r' \left( \varphi - \ln{r'} \right); \)
                
                    \textit{Вид уравнения:} \( F \left( \varphi, r, r' \right) = 0; \)

                    \textit{Характеристика уравнения:}
                        Полное неразрешенное относительно производной;

                    \textit{Общее решение:} \( r \cdot C^C = e^{C \varphi}. \)

                \item \( \tan{\frac{r}{r'}} = \ln{r}; \)
                
                    \textit{Вид уравнения:} \( F \left(r, r' \right) = 0; \)

                    \textit{Характеристика уравнения:}
                        Неразрешенное относительно производной, не содержащее аргумента;

                    \textit{Общее решение:} \( \ln{r} \cdot \arctg{\left( \ln r \right)} = \varphi + \frac{1}{2}\ln{\Big(\ln^2{r} + 1\Big)} + C. \)
                

                \item \( r = \frac{3}{2} \varphi r' + e^{r'}; \)
                
                    \textit{Вид уравнения:} \( F \left( \varphi, r, r' \right) = 0; \)

                    \textit{Характеристика уравнения:}
                        Уравнение Лагранжа;

                    \textit{Общее решение:} 
                        \(
                            \left\lbrace
                                \begin{aligned}
                                    r &= \frac{3}{2}\varphi p + e^p, \\
                                    \varphi &= \frac{C - \left( 2p^2 - 4p + 4 \right)e^p}{p^3}.
                                \end{aligned}
                            \right.    
                        \)

                \item \( \dot{x}^2 - 2x \dot{x} = x^2 \cdot \left( e^{2t} - 1 \right); \)
                
                    \textit{Вид уравнения:} \( F \left( t, x, \dot{x} \right) = 0; \)

                    \textit{Характеристика уравнения:}
                        Полное неразрешенное относительно производной;

                    \textit{Общее решение:} \( \ln{x} = t \pm e^t + C. \)

                
                \item \( \ln{\theta} = \ln{r'} + r'^2 - 1; \)
                
                    \textit{Вид уравнения:} \( F \left( \theta, r' \right) = 0; \)

                    \textit{Характеристика уравнения:}
                        Неразрешенное относительно производной, не содержащее функции;

                    \textit{Общее решение:}
                        \(
                            \left\lbrace
                                \begin{aligned}
                                    \theta &= p e^{p^2-1}, \\
                                    r &= \left( p^2 - \frac{1}{2} \right) e^{p^2-1} + C.
                                \end{aligned}
                            \right.    
                        \)

            \end{enumerate}

            
    \pagebreak

    \section{Задание 2}
        \subsection{Постановка задачи}
            Разрешить следующие уравнения относительно производной и, используя метод Эйлера, найти значение функции в точке. Нарисовать график искомой функции. Реализацию решения проводить на языке <<C++>>:

            \begin{enumerate}
                \item \( \sec^2{(1 -y- x)} = y'^2 - \tan{xy} + 2; \quad y \left( \frac{\pi}{4} \right) = 1, ~ y \left( \frac{\pi}{3} \right) = ?;  \)
                \item \( e^{x-y} = \cos\big( y' \sin{x} - \tan^2(\sec{xy}) - \tan{y} \big); \quad y \left( \frac{\pi}{3} \right) = \ln 7, ~ y(1) = ?. \)
            \end{enumerate}

        
        \subsection{Решение}
            \begin{enumerate}
                \item \( \sec^2{(1 -y- x)} = y'^2 - \tan{xy} + 2; \quad y \left( \frac{\pi}{4} \right) = 1, ~ y \left( \frac{\pi}{3} \right) = ?;  \)
                    \label{eq1}

                    \textit{Разрешённое уравнение:}
                        \( y' = \pm \sqrt{\sec^2 \left( 1 - y - x \right) + \tan{xy} - 2}; \)

                    \textit{Значение функции:}
                        \(  \left[
                            \begin{aligned}
                                &y_1\left( \frac{\pi}{3} \right) \approx 1.85955\dots, \\
                                &y_2\left( \frac{\pi}{3} \right) \approx 50.73742\dots;
                            \end{aligned}
                        \right. \)

                    \begin{figure}[H]
                        \centering
                        \includegraphics[width=10cm]{pictures/graph2_1.pdf}
                        \caption{График решений уравнения (\ref{eq1})}
                    \end{figure}

                    \pagebreak
                    \lstinputlisting[language=C++,
                        caption=Код программы,
                        style=Cpps,
                        basicstyle=\footnotesize\dejavu,
                        frame=lines]{code/2_1.cpp}

                \pagebreak
                

                \item \( e^{x-y} = \cos\big( y' \sin{x} - \tan^2(\sec{xy}) - \tan{y} \big); \quad y \left( \frac{\pi}{3} \right) = \ln 7, ~ y(1) = ?; \)
                    \label{eq2}

                    \textit{Разрешённое уравнение:}
                        \( y' = \frac{\arccos \left( e^{x-y} \right) + \tan^2 ( \sec{xy} ) + \tan y}{\sin x}; \)

                    \textit{Значение функции:}
                        \( y\left( 1 \right) \approx 1.97059\dots;\)

                    \begin{figure}[H]
                        \centering
                        \includegraphics[width=10cm]{pictures/graph2_2.pdf}
                        \caption{График решений уравнения (\ref{eq2})}
                    \end{figure}

                    \pagebreak
                    \lstinputlisting[language=C++,
                        caption=Код программы,
                        style=Cpps,
                        basicstyle=\footnotesize\dejavu,
                        frame=lines]{code/2_2.cpp}

            \end{enumerate}



    \pagebreak

    \section{Задание 3}
        \subsection{Постановка задачи}
            Для следующих дифференциальных уравнений определить тип, дать характеристику и найти общее решение с помощью программ компьютерной математики:

            \begin{enumerate}
                \item \( y'' \cdot \cos{y} = y'^2 \cdot \cot{y}; \)
                \item \( u^2 + 4tu \dot{u} + t^2 \dot{u}^2 + t^2 u \ddot{u} = 2tu \left( u + t \dot{u} \right) \cdot \tan{t}; ~ [z = \tan{t}]; \)
                \item \( \frac{\ddot{x}}{\dot{x}^2 + 1} = \dot{x}. \)
            \end{enumerate}

        
        \subsection{Решение}
            \begin{enumerate}
                \item \( y'' \cdot \cos{y} = y'^2 \cdot \cot{y}; \)
                    
                    \textit{Тип уравнения:} 
                        Допускающее интегрирование;

                    \textit{Характеристика уравнения:}
                        Вполне интегрируемое уравнение, с интегрирующим множителем \( \frac{\cos{y} + 1}{\sin{y}\cos{y}} \);

                    \textit{Общее решение:} \( \sin^2{y} = C_1 e^{C_2 x} \left( \cos{y}+1 \right). \)

                \item \( u^2 + 4tu \dot{u} + t^2 \dot{u}^2 + t^2 u \ddot{u} = 2tu \left( u + t \dot{u} \right) \cdot \tan{t}; ~ [z = \tan{t}]; \)
                
                    \textit{Тип уравнения:}
                        Допускающее интегрирование;

                    \textit{Характеристика уравнения:}
                        Вполне интегрируемое уравнение с заменой \( z = \tan{t} \);

                    \textit{Общее решение:} \( t^2u^2 = C_1 \tan{t} + C_2. \)


                \pagebreak

                \item \( \frac{\ddot{x}}{\dot{x}^2 + 1} = \dot{x}; \)
                                
                    \textit{Тип уравнения:}
                        Допускающее интегрирование;

                    \textit{Характеристика уравнения:}
                        Вполне интегрируемое уравнение;

                    \textit{Общее решение:} \( \sin\left(x + C_1\right) = C_2 e^t. \)
                    


            \end{enumerate}


    \pagebreak

    \section{Заключение}
        В этой лабораторной работе мы решили дифференциальные уравнения, неразрешённые относительно производной, и уравнения высших порядков, нашли значения функций и построили их графики с помощью производной.

\end{document}