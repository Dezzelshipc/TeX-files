\documentclass[14pt, a4paper, titlepage, fleqn]{extarticle}

\usepackage[russian]{babel}

\usepackage{amsmath}
\usepackage{amssymb}
\usepackage{listings}
\usepackage{color}

\lstset{  
    morekeywords=[1]{
        define, define-syntax, define-macro, lambda, define-stream, stream-lambda
    },
    morekeywords=[2]{
        begin, call-with-current-continuation, call/cc,
        call-with-input-file, call-with-output-file, case, cond,
        do, else, for-each, if,
        let*, let, let-syntax, letrec, letrec-syntax,
        let-values, let*-values,
        and, or, not, delay, force,
        quasiquote, quote, unquote, unquote-splicing,
        map, fold, syntax, syntax-rules, eval, environment, query
    },
    morekeywords=[3]{
        import, export
    },
    alsodigit=!\$\%&*+-./:<=>?@^_~,
    sensitive=true,
    morecomment=[l]{;},
    morecomment=[s]{\#|}{|\#},
    morestring=[b]",
    basicstyle=\small\ttfamily,
    keywordstyle=\bf\ttfamily\color[rgb]{0,.3,.7},
    commentstyle=\color[rgb]{0.133,0.545,0.133},
    stringstyle={\color[rgb]{0.75,0.49,0.07}},
    upquote=true,
    breaklines=true,
    breakatwhitespace=true,
    literate=*{`}{{`}}{1},
    showstringspaces=false,
    language=Go
}

\newcommand{\rnc}[1]
    {\MakeUppercase{\romannumeral #1}}

\title{Лабораторная работа №1 по дисциплине <<Дифференциальные уравнения>>}
\author{Держапольский Юрий Витальевич}
% \date{}

\begin{document}

    \maketitle

    \tableofcontents

    \pagebreak

    \section{Введение}
        В этой лабораторной работе мы научимся решать дифференциальные
        уравнения и верстать их в \LaTeX.

    \pagebreak

    \section{Задание 1: Вычислить неопределённый интеграл}
        \subsection{Постановка задачи}
            Найти следующий интеграл с подробным описанием всех действий:
            \[ \int \sin{ \sqrt[3]{t+1} } ~ dt. \]
        
        \subsection{Решение}
        \[
            \begin{split}
                \int \sin{ \sqrt[3]{t+1} } ~ dt &= \circledast
                \begin{vmatrix}
                    \sqrt[3]{t+1} = x \\
                    t = x^3 - 1 \\
                    dt = 3x^2dx 
                \end{vmatrix}
                \circledast = \int 3x^2\sin{x} ~ dx = \\
                &= -3 \int x^2 ~ d(\cos{x}) = \\
                &= -3 \left( 
                    x^2\cos{x} - \int \cos{x} ~ d \left(x^2\right) 
                \right) = \\
                &= -3x^2\cos{x} + 6\left(\int x \cos{x} ~ dx\right) = \\
                &= -3x^2\cos{x} + 6\left(\int x ~ d(\sin{x})\right) = \\
                &= -3x^2\cos{x} + 6\left(x\sin{x} - \int \sin{x} ~ dx \right)=\\
                &= -3x^2\cos{x} + 6x\sin{x} + 6\cos{x} + C =\\
                &= 6 \sqrt[3]{t+1} \sin{\sqrt[3]{t+1}} + \\
                &+ 3\left(2 - (t+1)^{\frac{2}{3}}\right) \cos{\sqrt[3]{t+1}}+C.
            \end{split}    
        \]

        \textit{Ответ:} 
        \[
            \displaystyle
            \int \sin{ \sqrt[3]{t+1} } ~ dt
            = 6 \sqrt[3]{t+1} \sin{\sqrt[3]{t+1}}
            + 3\left( 2 - (t+1)^{\frac{2}{3}} \right) \cos{\sqrt[3]{t+1}} + C.
        \]

    \pagebreak

    \section{Задание 2: Численно вычислить интеграл}
        \subsection{Постановка задачи}
            Четырьмя методами численно вычислить следующий интеграл
            с точностью \( \varepsilon = 10^{-4} \).
            Реализацию решения проводить на языке <<Go>>:
            
            \[ \int\limits_0^1 \frac{\ln(1-t)}{t^2+1} ~ dt. \]

        \subsection{Решение}
            \textit{Точное значение:}
            \( 
                \displaystyle
                \int\limits_0^1 \frac{\ln(1-t)}{t^2+1} ~ dt \approx -0.643767\dots
            \)

            \begin{enumerate}
                \item Метод левых прямоугольников:
                    \[
                        \int\limits_a^b f(x) ~ dx \approx
                        \sum_{k=0}^{n-1} f\left( x_k \right) \cdot \Delta x,
                        \quad x_k \in [a, b], ~ \Delta x = \frac{b-a}{n}. 
                    \]

                    \textit{Найденное значение:}
                    \[
                        \int\limits_0^1 \frac{\ln(1-t)}{t^2+1} ~ dt \approx
                        -0.643667\dots, ~ n = 30291,
                    \]
                    \[
                        \vert-0.643767+0.643667\vert = 9.999 
                        \cdot 10^{-5} < \varepsilon.
                    \]

                    \pagebreak

                    \textit{Код программы:}
                    \begin{lstlisting}[
                        language=Go,
                        basicstyle=\footnotesize\ttfamily,
                        frame=lines
                    ]
package main

import "math"
import "fmt"

func f(x float64) float64 {
    return ((math.Log(1-x))/(1+x*x))
}

func left_rect(n int, a float64, b float64) float64 {
    s := 0.0
    delta := (b-a)/ float64(n)
    for i := 0; i < n; i++ {
        s += f(a + delta * float64(i))
    }
    s *= delta
    return s
}

func main() {
    n := 0
    eps := 1e-4
    a := 0.0
    b := 1.0
    real := -0.643767
    s:= 10000
    for math.Abs(real - s) >= eps {
        n++
        s = left_rect(n,a,b)
    }
    fmt.Println(s, n, math.Abs(real-s))
}
                    \end{lstlisting}

                    \pagebreak

                \item Метод правых прямоугольников:
                    \[
                        \int\limits_a^b f(x) ~ dx \approx
                        \sum_{k=1}^{n} f\left( x_k \right) \cdot \Delta x,
                        \quad x_k \in [a, b], ~ \Delta x = \frac{b-a}{n}.  
                    \]


                    \textit{Найденное значение:}
                    \[
                        \int\limits_0^1 \frac{\ln(1-t)}{t^2+1} ~ dt \approx
                        -0.643866\dots, ~ n = 27507,
                    \]
                    \[
                        \vert-0.643767+0.643866\vert = 9.999 
                        \cdot 10^{-5} < \varepsilon.
                    \]

                    \pagebreak
                    \textit{Код программы:}
                    \begin{lstlisting}[
                        language=Go,
                        basicstyle=\footnotesize\ttfamily,
                        frame=lines
                    ]
package main

import "math"
import "fmt"

func f(x float64) float64 {
    return ((math.Log(1-x))/(1+x*x))
}

func right_rect(n int, a float64, b float64) float64 {
    s := 0.0
    delta := (b-a)/ float64(n)
    eps := 1e-5
    for i := 1; i <= n; i++ {
        if (i != n) {
            s += f(a + delta * float64(i))
        } else {
            s += f(a + delta * float64(i) - eps)
        }
    }
    s *= delta
    return s
}

func main() {
    n := 0
    eps := 1e-4
    a := 0.0
    b := 1.0
    real := -0.643767
    s := 10000
    n=0
    for math.Abs(real - s) >= eps {
        n++
        s = right_rect(n,a,b)
    }
    fmt.Println(s, n, math.Abs(real-s))
}
                    \end{lstlisting}
                    \pagebreak
                
                \item Метод центральных прямоугольников:
                    \[
                        \int\limits_a^b f(x) ~ dx \approx
                        \sum_{k=0}^{n-1} f\left( \frac{x_k + x_{k+1}}{2}\right)\cdot
                        \Delta x,
                        \quad x_k \in [a, b], ~ \Delta x = \frac{b-a}{n}.
                    \]

                    \textit{Найденное значение:}
                    \[
                        \int\limits_0^1 \frac{\ln(1-t)}{t^2+1} ~ dt \approx
                        -0.643667\dots, ~ n = 1726,
                    \]
                    \[
                        \vert-0.643767+0.643667\vert = 9.998 
                        \cdot 10^{-5} < \varepsilon.
                    \]

                    \pagebreak
                    \textit{Код программы:}
                    \begin{lstlisting}[
                        language=Go,
                        basicstyle=\footnotesize\ttfamily,
                        frame=lines
                    ]
package main

import "math"
import "fmt"

func f(x float64) float64 {
    return ((math.Log(1-x))/(1+x*x))
}

func center_rect(n int, a float64, b float64) float64 {
    s := 0.0
    delta := (b-a)/ float64(n)
    for i := 0; i < n; i++ {
        s += f((2*a + delta * float64(2*i + 1))/2)
    }
    s *= delta
    return s
}

func main() {
    n := 0
    eps := 1e-4
    a := 0.0
    b := 1.0
    real := -0.643767
    s := 10000
    n=0
    for math.Abs(real - s) >= eps {
        n++
        s = center_rect(n,a,b)
    }
    fmt.Println(s, n, math.Abs(real-s))
}
                    \end{lstlisting}
                
                    \pagebreak

                \item Метод трапеций:
                    \[
                        \int\limits_a^b f(x) ~ dx \approx
                        \sum_{k=0}^{n-1} 
                        \frac{f\left( x_k \right) + f\left(x_{k+1}\right)}{2}
                        \cdot \Delta x,
                        \quad x_k \in [a, b], ~ \Delta x = \frac{b-a}{n}.  
                    \]

                    \textit{Найденное значение:}
                    \[
                        \int\limits_0^1 \frac{\ln(1-t)}{t^2+1} ~ dt \approx
                        -0.643866\dots, ~ n = 3676,
                    \]
                    \[
                        \vert-0.643767+0.643866\vert = 9.996
                        \cdot 10^{-5} < \varepsilon.
                    \]

                    \pagebreak

                    \textit{Код программы:}
                    \begin{lstlisting}[
                        language=Go,
                        basicstyle=\footnotesize\ttfamily,
                        frame=lines
                    ]
package main

import "math"
import "fmt"

func f(x float64) float64 {
    return ((math.Log(1-x))/(1+x*x))
}

func trapezoid_rect(n int, a float64, b float64) float64 {
    s := 0.0
    delta := (b-a)/ float64(n)
    eps := 1e-5
    for i := 0; i < n; i++ {
        if (i != n - 1) {
            s += (f(a + delta * float64(i)) +
                 f(a + delta * float64(i+1)))/2
        } else {
            s += (f(a + delta * float64(i)) +
                 f(a + delta * float64(i+1) - eps))/2
        }
    }
    s *= delta
    return s
}

func main() {
    n := 0
    eps := 1e-4
    a := 0.0
    b := 1.0
    real := -0.643767
    s:= 10000
    n=0
    for math.Abs(real - s) >= eps {
        n++
        s = trapezoid_rect(n,a,b)
    }
    fmt.Println(s, n, math.Abs(real-s))
}
                \end{lstlisting}
            \end{enumerate}

    \pagebreak

    \section{Задание 3: Решить уравнения}
        \subsection{Постановка задачи}
            Для следующих дифференциальных уравнений определить тип и найти
            общее решение с помощью программ компьютерной математики:
            \begin{enumerate}
                \item 
                \(
                    \displaystyle
                    r' = -\frac{5\theta + 3r + 2}{3\theta -11r-6}.
                \)

                \item 
                \(
                    \displaystyle
                    \frac{1-\dot{u}}{1+\dot{u}} \tg (t-u-1) = 2t+2u+8.
                \)

                \item 
                \(
                    \displaystyle
                    \dot{y} = \frac{1}{t \cdot \cos{y} + \sin{2y}}.
                \)
                
                \item 
                \(
                    t\dot{u} - u^2=2u+1.
                \)
            \end{enumerate}

        \subsection{Решение}
            \begin{enumerate}
                \item 
                \(
                    \displaystyle
                    r' = -\frac{5\theta + 3r + 2}{3\theta -11r-6}. 
                \)

                \textit{Тип уравнения:}
                Уравнение в полных дифференциалах.

                \textit{Общее решение:}
                \(
                    \displaystyle
                    6 \theta r-12r-11r^2+5\theta^2+4\theta=C.
                \)

                \item 
                \(
                    \displaystyle
                    \frac{1-\dot{u}}{1+\dot{u}} \tg (t-u-1) = 2t+2u+8.
                \)

                \textit{Тип уравнения:}
                Уравнение в полных дифференциалах.

                \textit{Общее решение:}
                \(
                    \displaystyle
                    % -\ln{(\cos{(u-t+1)})}-u^2-2ut-8u=t^2+8t+C.
                    e^{(u+t+4)^2} \cos(t-u-1) = C.
                \)

                \pagebreak
                \item 
                \(
                    \displaystyle
                    \dot{y} = \frac{1}{t \cdot \cos{y} + \sin{2y}}.
                \)

                \textit{Тип уравнения:}
                Линейное приведённое неоднородное дифференциальное уравнение 
                \rnc{1} порядка с переменными коэффициентами относительно
                \(t = t(y)\).

                \textit{Общее решение:}
                \(
                    \displaystyle
                    t = C e^{\sin{y}} - 2 \sin{y}-2.
                \)
                
                \item 
                \(
                    \displaystyle
                    t\dot{u} - u^2=2u+1.
                \)

                \textit{Тип уравнения:} Уравнение с разделяющимися переменными.
                \textit{Общее решение:}
                \(
                    \displaystyle
                    -\frac{1}{u+1}=\ln{t}+C.
                \)
            \end{enumerate}

    \pagebreak

    \section{Заключение}
        Мы научились решать дифференциальные 
        уравнения и верстать их в \LaTeX.


\end{document}