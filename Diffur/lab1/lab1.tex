\documentclass[14pt, a4paper, titlepage, fleqn]{extarticle}

\usepackage[russian]{babel}

\usepackage{amsmath}
\usepackage{amssymb}
\usepackage{listings}

\title{Лабораторная работа №1 по дисциплине <<Дифференциальные уравненеия>>}
\author{Держапольский Юрий Витальевич}
% \date{}

\begin{document}

    \maketitle

    \tableofcontents

    \pagebreak

    \section{Введение}
        Веедение здесь

    \pagebreak

    \section{Задание 1: Вычислить неорпеделённый интеграл}
        \subsection{Постановка задачи}
            Найти следующий интеграл с подробным описанием всех действий:
            \[ \int \sin{ \sqrt[3]{t+1} } ~ dt \]
        
        \subsection{Решение}
        \[
            \begin{split}
                \int \sin{ \sqrt[3]{t+1} } ~ dt &= \circledast
                \begin{vmatrix}
                    \sqrt[3]{t+1} = x \\
                    t = x^3 - 1 \\
                    dt = 3x^2dx
                \end{vmatrix}
                \circledast = \int 3x^2\sin{x} ~ dx = \\
                &= -3 \int x^2 ~ d(\cos{x}) = \\
                &= -3 \left( 
                    x^2\cos{x} - \int \cos{x} ~ d \left(x^2\right) 
                \right) = \\
                &= -3x^2\cos{x} + 6\left(\int x \cos{x} ~ dx\right) = \\
                &= -3x^2\cos{x} + 6\left(\int x ~ d(\sin{x})\right) = \\
                &= -3x^2\cos{x} + 6\left(x\sin{x} - \int \sin{x} ~ dx \right)=\\
                &= -3x^2\cos{x} + 6x\sin{x} + 6\cos{x} + C =\\
                &= 6 \sqrt[3]{t+1} \sin{\sqrt[3]{t+1}} + 
                    3\left(2 - (t+1)^{\frac{2}{3}}\right) \cos{\sqrt[3]{t+1}}+C
            \end{split}    
        \]

        \textit{Ответ:} 
        \[
            \displaystyle
            \int \sin{ \sqrt[3]{t+1} } ~ dt
            = 6 \sqrt[3]{t+1} \sin{\sqrt[3]{t+1}}
            + 3\left( 2 - (t+1)^{\frac{2}{3}} \right) \cos{\sqrt[3]{t+1}} + C 
        \]

    \pagebreak

    \section{Задание 2: Численно вычислить интеграл}
        \subsection{Постановка задачи}
            Четыремя методами численно вычислить следующий интеграл
            с точностью \( \varepsilon = 10^{-4} \).
            Реализацию решения проводить на языке <<Go>>:
            
            \[ \int\limits_0^1 \frac{\ln(1-t)}{t^2+1} ~ dt \]

        \subsection{Решение}
            \textit{Точное значение:}
            \( 
                \displaystyle
                \int\limits_0^1 \frac{\ln(1-t)}{t^2+1} ~ dt \approx -0.643767
            \)

            \begin{enumerate}
                \item Метод левых прямоугольников
                    \[
                        \int\limits_a^b f(x) ~ dx \approx
                        \sum_{k=0}^{n-1} f\left( x_k \right) \cdot \Delta x,
                        \quad x_k \in [a, b], ~ \Delta x = \frac{b-a}{n}    
                    \]

                    \textit{Найденное значение:}
                    \[
                        \int\limits_0^1 \frac{\ln(1-t)}{t^2+1} ~ dt \approx
                    \]

                    \textit{Код программы:}
                    \begin{lstlisting}[
                        language=Go,
                        basicstyle=\footnotesize\ttfamily,
                        frame=lines
                    ]
asdasdasd
                    \end{lstlisting}


                \item Метод правых прямоугольников
                \[
                    \int\limits_a^b f(x) ~ dx \approx
                    \sum_{k=1}^{n} f\left( x_k \right) \cdot \Delta x,
                    \quad x_k \in [a, b], ~ \Delta x = \frac{b-a}{n}    
                \]

                \textit{Найденное значение:}
                \[
                    \int\limits_0^1 \frac{\ln(1-t)}{t^2+1} ~ dt \approx
                \]

                \textit{Код программы:}
                \begin{lstlisting}[
                    language=Go,
                    basicstyle=\footnotesize\ttfamily,
                    frame=lines
                ]
asdasdasd
                \end{lstlisting}
                
                \item Метод центральных прямоугольников
                \[
                    \int\limits_a^b f(x) ~ dx \approx
                    \sum_{k=0}^{n-1} f\left( \frac{x_k + x_{k+1}}{2}\right)\cdot
                    \Delta x,
                    \quad x_k \in [a, b], ~ \Delta x = \frac{b-a}{n}    
                \]

                \textit{Найденное значение:}
                \[
                    \int\limits_0^1 \frac{\ln(1-t)}{t^2+1} ~ dt \approx
                \]

                \textit{Код программы:}
                \begin{lstlisting}[
                    language=Go,
                    basicstyle=\footnotesize\ttfamily,
                    frame=lines
                ]
asdasdasd
                \end{lstlisting}
                
                \item Метод трапеций
                \[
                    \int\limits_a^b f(x) ~ dx \approx
                    \sum_{k=0}^{n-1} 
                    \frac{f\left( x_k \right) + f\left(x_{k+1}\right)}{2}
                    \cdot \Delta x,
                    \quad x_k \in [a, b], ~ \Delta x = \frac{b-a}{n}    
                \]

                \textit{Найденное значение:}
                \[
                    \int\limits_0^1 \frac{\ln(1-t)}{t^2+1} ~ dt \approx
                \]

                \textit{Код программы:}
                \begin{lstlisting}[
                    language=Go,
                    basicstyle=\footnotesize\ttfamily,
                    frame=lines
                ]
asdasdasd
                \end{lstlisting}
            \end{enumerate}

    \pagebreak

    \section{Задание 3: Решить уравнения}
        \subsection{Постановка задачи}
            Для следующих дифференциальных уравнений определить тип и найти
            общее решение с помощью программ компьютерной математики:
            \begin{enumerate}
                \item 
                \(
                    \displaystyle
                    r' = -\frac{5\theta + 3r + 2}{3\theta -11r-6}    
                \)

                \item 
                \(
                    \displaystyle
                    \frac{1-\dot{u}}{1+\dot{u}} \tg (t-u-1) = 2t+2u+8
                \)

                \item 
                \(
                    \displaystyle
                    \dot{y} = \frac{1}{t \cdot \cos{y} + \sin{2y}}
                \)
                
                \item 
                \(
                    \displaystyle
                    t\dot{u} - u^2=2u+1    
                \)
            \end{enumerate}

        \subsection{Решение}
            \begin{enumerate}
                \item 
                \(
                    \displaystyle
                    r' = -\frac{5\theta + 3r + 2}{3\theta -11r-6}    
                \)

                \textit{Тип уравнения:}
                Обычное дифференциальное уравнение 1 порядка

                \textit{Общее решение:}
                \(
                    \displaystyle
                    r(\theta) = \frac{3 (\theta - 2)}{11} \pm
                    \frac{i \sqrt{C - 2 (\frac{5 \theta^2}{2} + 2 \theta) - 
                    \frac{9}{11} (\theta - 2)^2}}{\sqrt{11}}
                \)

                \item 
                \(
                    \displaystyle
                    \frac{1-\dot{u}}{1+\dot{u}} \tg (t-u-1) = 2t+2u+8
                \)

                \textit{Тип уравнения:}
                Обычное дифференциальное уравнение 1 порядка (?)

                \textit{Общее решение:}
                \(
                    \displaystyle
                \)

                \item 
                \(
                    \displaystyle
                    \dot{y} = \frac{1}{t \cdot \cos{y} + \sin{2y}}
                \)

                \textit{Тип уравнения:}
                Обычное дифференциальное уравнение 1 порядка

                \textit{Общее решение:}
                \(
                    \displaystyle
                    y(t) = \pm \arcsin \left(\frac{1}{2} \left(
                        \mp 2 W \left(e^{-\frac{t}{2} - 1} C \right) \pm t \pm 2
                        \right)\right)
                \)
                
                \item 
                \(
                    \displaystyle
                    t\dot{u} - u^2=2u+1    
                \)

                \textit{Тип уравнения:} Уравнение с разделяющимися переменными.
                \textit{Общее решение:}
                \(
                    \displaystyle
                    u = -\frac{\ln(t) + 1 + C}{\ln(t) + C}
                \)
            \end{enumerate}

    \pagebreak

    \section{Заключение}
    aslhdakjshdkjsadasdsd


\end{document}